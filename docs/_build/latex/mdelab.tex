%% Generated by Sphinx.
\def\sphinxdocclass{report}
\documentclass[letterpaper,10pt,french]{sphinxmanual}
\ifdefined\pdfpxdimen
   \let\sphinxpxdimen\pdfpxdimen\else\newdimen\sphinxpxdimen
\fi \sphinxpxdimen=.75bp\relax

\PassOptionsToPackage{warn}{textcomp}
\usepackage[utf8]{inputenc}
\ifdefined\DeclareUnicodeCharacter
% support both utf8 and utf8x syntaxes
  \ifdefined\DeclareUnicodeCharacterAsOptional
    \def\sphinxDUC#1{\DeclareUnicodeCharacter{"#1}}
  \else
    \let\sphinxDUC\DeclareUnicodeCharacter
  \fi
  \sphinxDUC{00A0}{\nobreakspace}
  \sphinxDUC{2500}{\sphinxunichar{2500}}
  \sphinxDUC{2502}{\sphinxunichar{2502}}
  \sphinxDUC{2514}{\sphinxunichar{2514}}
  \sphinxDUC{251C}{\sphinxunichar{251C}}
  \sphinxDUC{2572}{\textbackslash}
\fi
\usepackage{cmap}
\usepackage[T1]{fontenc}
\usepackage{amsmath,amssymb,amstext}
\usepackage{babel}



\usepackage{times}
\expandafter\ifx\csname T@LGR\endcsname\relax
\else
% LGR was declared as font encoding
  \substitutefont{LGR}{\rmdefault}{cmr}
  \substitutefont{LGR}{\sfdefault}{cmss}
  \substitutefont{LGR}{\ttdefault}{cmtt}
\fi
\expandafter\ifx\csname T@X2\endcsname\relax
  \expandafter\ifx\csname T@T2A\endcsname\relax
  \else
  % T2A was declared as font encoding
    \substitutefont{T2A}{\rmdefault}{cmr}
    \substitutefont{T2A}{\sfdefault}{cmss}
    \substitutefont{T2A}{\ttdefault}{cmtt}
  \fi
\else
% X2 was declared as font encoding
  \substitutefont{X2}{\rmdefault}{cmr}
  \substitutefont{X2}{\sfdefault}{cmss}
  \substitutefont{X2}{\ttdefault}{cmtt}
\fi


\usepackage[Sonny]{fncychap}
\ChNameVar{\Large\normalfont\sffamily}
\ChTitleVar{\Large\normalfont\sffamily}
\usepackage{sphinx}

\fvset{fontsize=\small}
\usepackage{geometry}


% Include hyperref last.
\usepackage{hyperref}
% Fix anchor placement for figures with captions.
\usepackage{hypcap}% it must be loaded after hyperref.
% Set up styles of URL: it should be placed after hyperref.
\urlstyle{same}

\addto\captionsfrench{\renewcommand{\contentsname}{Work:}}

\usepackage{sphinxmessages}
\setcounter{tocdepth}{1}



\title{MDE LAB}
\date{août 18, 2020}
\release{}
\author{Equipe MDE}
\newcommand{\sphinxlogo}{\vbox{}}
\renewcommand{\releasename}{}
\makeindex
\begin{document}

\ifdefined\shorthandoff
  \ifnum\catcode`\=\string=\active\shorthandoff{=}\fi
  \ifnum\catcode`\"=\active\shorthandoff{"}\fi
\fi

\pagestyle{empty}
\sphinxmaketitle
\pagestyle{plain}
\sphinxtableofcontents
\pagestyle{normal}
\phantomsection\label{\detokenize{index::doc}}


16907
\begin{itemize}
\item {} 
\sphinxstylestrong{18 juin point dossier Danyl de 10h à 12h (achats et proposition mise en avant du lab sur le site de la MDE)}

\item {} 
19 juin 10h à 12h réunion secteur numérique sur les grands dossiers et priorités de chacun

\item {} 
22 juin réunion de service toute la médiathèque 9h 30 à 12h

\item {} 
congé lundi 13 juillet
\begin{itemize}
\item {} 
Validation autres congés

\end{itemize}

\item {} 
25/09 9h 30 à 17h Ateliers Lab semi\sphinxhyphen{}ruralité 9h 30 à 12h30 et grande ruralité 14h à 17h à Chamarande \sphinxhyphen{} Tout de la Gruue Jaune

\item {} 
15/09 et 16/09 Ateliers Lab semi\sphinxhyphen{}ruralité et grande ruralité

\item {} 
15/09 et 16/09 Ateliers Lab semi\sphinxhyphen{}ruralité et grande ruralité

\item {} 
25/09 et 26/09 Ateliers Lab semi\sphinxhyphen{}ruralité et grande ruralité

\item {} 
21/10 et 22/10 Ateliers Lab semi\sphinxhyphen{}ruralité et grande ruralité

\item {} 
01/10 et 02/10 CNFPT Evry Communications et relations professionnelles

\item {} 
02/11 et 03/11 CNFPT Evry Organisation et fonctionnement de la fonction publqiue territoriale

\item {} 
12/11 Cotech Lab à la MDE de 12h à 14h

\end{itemize}

Achats \sphinxurl{https://docs.google.com/spreadsheets/d/1hJBHDOaEBfuzhKLhrwd5dCQWXbcdmFH2tYDBDMaE7wc/edit?usp=sharing}

Liens Aurélia: \sphinxurl{https://docs.google.com/spreadsheets/d/1ynwj1Z0iR1U8yI0ZfISqASkqrM4zpbeJJJEpm561e2k/edit?usp=sharing\_eil\&ts=5ea6eaeb}

TODO:

\begin{sphinxadmonition}{important}{Important:}\begin{itemize}
\item {} 
VOIR MANUTAN TRANSPORT AVANT MERCREDI 27: \sphinxurl{https://www.manutan-collectivites.fr/}

\item {} 
réfléchir présentation lab 15 juin

\item {} 
Rédiger tuto 3Doodler

\item {} 
Tuto 3d Sense

\item {} 
Utiliser vinyl Curio

\item {} 
Utiliser deux outils en même temps sur la Curio

\item {} 
Importer SVG dans Curio

\item {} 
voir société qui recycle le filament

\item {} 
Faire tableau factures machines achetés (3D UpBox + Epson SC\sphinxhyphen{}P600 + Sense 3D V2), date d’achat et durée de garantie

\end{itemize}
\end{sphinxadmonition}

Chaque jour:

\begin{sphinxadmonition}{important}{Important:}\begin{itemize}
\item {} 
lire un chapitre Espaces de création numérique en bibiliothèque et faire une synthèse

\end{itemize}
\end{sphinxadmonition}

Voir atelier papercraft

S’inscrire formation gestion groupe cnfpt et écoute active.
Voir descendre caissons (plateforme élévatrce), diable, chariot à tiroir.
\begin{itemize}
\item {} 
chercher type filament 3doodler

\item {} 
ramener backup Dirt Rally

\item {} 
Description lab: \sphinxurl{https://laboratoirecreation.banq.qc.ca/index.php?title=Accueil\&fbclid=IwAR3gQlc4zd3KdpDU9Bf5OI7psD25HXGawJW-1\_2wad8Zvsbxdi3-uRuUPYc}

\item {} 
A lire: \sphinxurl{https://laboratoirecreation.banq.qc.ca/index.php/Pourquoi\_des\_laboratoires\_de\_cr\%C3\%A9ation\_dans\_les\_biblioth\%C3\%A8ques\%3F}

\item {} 
Broderie 3D: \sphinxurl{https://youtu.be/RC4Dsh3-Wis}

\end{itemize}

DONE:
\begin{itemize}
\item {} 
contacter a4 et tiertime avec vidéo problème 3D Up Box

\end{itemize}

Best Free Alternatives To Adobe Creative Suite: \sphinxurl{https://www.youtube.com/watch?v=o0Bvr070N-k}


\bigskip\hrule\bigskip


\begin{sphinxadmonition}{note}{Note:}
Vous trouverez ici la documentation sur l’utilisation du matériel de la MDE.

Pour toute(s) question(s), suggestion(s), demande(s), vous pouvez nous écrire directement à cette adresse email: \sphinxhref{mailto:numerique-mde@cd-essonne.fr}{numerique\sphinxhyphen{}mde@cd\sphinxhyphen{}essonne.fr}
\end{sphinxadmonition}


\bigskip\hrule\bigskip


\noindent{\hspace*{\fill}\sphinxincludegraphics{{essonne}.png}}

\noindent{\hspace*{\fill}\sphinxincludegraphics{{mde}.png}}


\chapter{Idées d’ateliers}
\label{\detokenize{tutorials/work/ideas/index:idees-d-ateliers}}\label{\detokenize{tutorials/work/ideas/index::doc}}\begin{itemize}
\item {} 
\sphinxhref{https://youtu.be/NsAs2pYi6No}{Fabriquer sa console de jeu}

\end{itemize}


\section{Petit projets}
\label{\detokenize{tutorials/work/ideas/index:petit-projets}}\begin{itemize}
\item {} 
\sphinxhref{https://wikifab.org/wiki/Vid\%C3\%A9oprojecteur\_pour\_smartphone}{Vidéoprojecteur pour smartphone}

\item {} 
\sphinxhref{https://wikifab.org/wiki/Thermoformage}{Thermoformage}

\end{itemize}


\section{Gros projets}
\label{\detokenize{tutorials/work/ideas/index:gros-projets}}\begin{itemize}
\item {} 
\sphinxhref{https://wikifab.org/wiki/Babyfoot\_6\_personnes}{Babyfoot}

\item {} 
\sphinxhref{https://wikifab.org/wiki/Fabrication\_D\%27une\_Borne\_D\%27arcade}{Borne d’arcade}

\item {} 
\sphinxhref{https://wikifab.org/wiki/Vikart\_:\_kart\_\%C3\%A9lectrique\_pas\_cher,\_performant\_et\_facile\_\%C3\%A0\_construire}{Kart électrique}

\item {} 
\sphinxhref{https://wikifab.org/wiki/Voiture\_t\%C3\%A9lecommand\%C3\%A9e\_en\_bluetooth\_par\_son\_smartphone}{Voiture télécommandée par smartphone}

\end{itemize}


\chapter{Achats}
\label{\detokenize{tutorials/work/achats/index:achats}}\label{\detokenize{tutorials/work/achats/index::doc}}
\begin{sphinxadmonition}{note}{Note:}
Dimmensions  en mm (longueur, largeur, hauteur).
\begin{itemize}
\item {} 
3D Up Box+: 490 x 490 x 510

\item {} 
Silhouette Curio: 290 x 490 x 250

\item {} 
Roland GS\sphinxhyphen{}24: 500 x 870 x 1070

\item {} 
Bernina: 460 x 760 x 400

\item {} 
Dagoma: 330 x 330 x 550

\end{itemize}

Max: 500 x 870 x 550 (pieds Roland non calculés)
\end{sphinxadmonition}

Explications de ma suggestion: le matériel est transporté dans des bacs placés éventuellement dans une caisse de transport (ou deux). La caisse de transport s’il y en a une est maintenue par une sangle ou plusieurs d’une longueur de 5 mètres pour être sûr de bien pouvoir faire le tour avec.
Il n’est pas nécessaire de pouvoir faire rentrer toutes les machines en même temps, car elles ne seront pas toutes déployées au même endroit.
La caisse de transport est optionnelle, car la surface du véhicule est antidérapante, donc les bacs devraient suffire à les maintenir en place.
Une fois arrivé il faut sortir le charriot à deux plateaux avec rebords du véhicule, enlever les sangles de la caisse de transport pour l’ouvrir, prendre les bacs et les mettre sur le charriot.
Les objets lourds et volumineux comme l’imprimante 3D vont sur le plateau du dessus pour éviter d’avoir à me baisser et parce qu’il n’y a pas de limite de hauteur.

\begin{sphinxadmonition}{note}{Note:}
Pour mieux organiser le transport il faudra faire une liste du matériel nécessaire pour la réalisation de chaque atelier et du matériel de secours.
\end{sphinxadmonition}


\section{Matériel indispensable}
\label{\detokenize{tutorials/work/achats/index:materiel-indispensable}}
Chariot 1 plateau bois rebords 965x655mm 250 Kg Variofit (entre deux plateaux 535 mm): \sphinxurl{https://www.manutan-collectivites.fr/chariot-1-plateau-bois-rebords-965x655mm-250-kg-variofit-itg3048277.html?refCom=Tous\&webdo\_20000509=Tous\&webdo\_20000225=Tous\&webdo\_20003721=Tous}

\begin{sphinxadmonition}{note}{Note:}
Les caractéristiques importantes sont:
\begin{itemize}
\item {} 
les dimensions pour pouvoir transporter toutes les machines (pas forcément toutes à la fois)

\item {} 
avoir deux plateaux afin d’en avoir un en hauteur pour éviter de me baisser

\item {} 
avoir des roues blocables pour les transports

\item {} 
avoir des rebords surtout sur le plateau du haut pour éviter de faire tomber les accessoires

\item {} 
disposer d’un frein pour éviter une perte de contrôle dans une descente

\item {} 
avoir une poignée en hauteur

\end{itemize}
\end{sphinxadmonition}

\begin{sphinxadmonition}{note}{Note:}
Les roues me paraissent avoir un diamètre légèrement supérieur à la moyenne sur le charriot sélectionné, cela permettrait de l’utiliser sur de la terre.
\end{sphinxadmonition}

\noindent\sphinxincludegraphics{{chariot}.png}

2x (3D Up Box+ + un en plus): Bac 175L, parois et fond pleins , recyclé, Noir LxlxH = 800x600x425mm: \sphinxurl{https://www.manutan-collectivites.fr/bac-175l-parois-et-fond-pleins-recycle-noir-lxlxh-800x600x425mm-110267003.html?refCom=Tous\&webdo\_20001652=Tous\&webdo\_20000041=Tous\&webdo\_20003804=Tous}

\begin{sphinxadmonition}{note}{Note:}
Les caractéristiques importantes sont:
\begin{itemize}
\item {} 
les dimensions (à noter que le bac fait presque la hauteur de la 3D Up Box afin de la protéger et saisir facilement)

\item {} 
des poignets

\end{itemize}

Peut\sphinxhyphen{}être, voir s’il serait possible d’avoir le bac avec l’imprimante 3D dedans directement sur le plateau du dessus du charriot et maintenu par les sangles pour ne pas avoir à la soulever.
\end{sphinxadmonition}

\noindent\sphinxincludegraphics{{bac_grand}.png}

7x (Curio + Bernina + Dagoma 1 + Dagoma 2 + onduleurs + 10 PCs + 10 tablettes graphiques): Bac 22L, parois et fond pleins , recyclé, Noir LxlxH = 600x400x120mm: \sphinxurl{https://www.manutan-collectivites.fr/bac-22l-parois-et-fond-pleins-recycle-noir-lxlxh-600x400x120mm-110266003.html?refCom=Tous\&webdo\_20001652=Tous\&webdo\_20000041=Tous\&webdo\_20003804=Tous}

\noindent\sphinxincludegraphics{{bac_moyen}.png}


\section{Matériel optionnel}
\label{\detokenize{tutorials/work/achats/index:materiel-optionnel}}
\begin{sphinxadmonition}{note}{Note:}
Ci\sphinxhyphen{}dessous les propositions sont optionnelles, le but étant aussi de perdre le moins de temps possible pour déployer le matériel.
\end{sphinxadmonition}

Cembox de transport 750 litres (1700 x 840 x 800): \sphinxurl{https://www.manutan-collectivites.fr/cembox-750-itg3421037.html?refCom=Tous\&webdo\_20000225=Tous\&webdo\_20000041=Tous\&webdo\_20001652=Tous}

\begin{sphinxadmonition}{note}{Note:}
La caisse est longue et fait 800 mm de haut, il sera possible de superposer des accessoires et câbles au\sphinxhyphen{}dessus des machines.

La caisse de transport ne sera pas déplacée, de plus elle pèse 42 kg à vide.
\end{sphinxadmonition}

\noindent\sphinxincludegraphics{{caisse}.png}

Sangle d’arrimage à tendeur cliquet \sphinxhyphen{} Force 350 kg \sphinxhyphen{} Manutan (5 mètres): \sphinxurl{https://www.manutan-collectivites.fr/sangle-d-arrimage-a-tendeur-cliquet-force-350-kg-manutan-itg2259295.html}

\begin{sphinxadmonition}{note}{Note:}
5 mètres me semble être la taille minimale pour faire le tour de la caisse ou du charriot. Il faut voir combien il en faut.
\end{sphinxadmonition}

\noindent\sphinxincludegraphics{{sangle}.png}


\section{A voir}
\label{\detokenize{tutorials/work/achats/index:a-voir}}
\noindent\sphinxincludegraphics{{caisses}.png}

Caisse de transport en contreplaqué avec couvercle: \sphinxurl{https://www.manutan-collectivites.fr/caisse-de-transport-en-contreplaque-avec-couvercle-cf-2169516.html?q=caisse+de+transport+en+contreplaqu\%C3\%A9}


\chapter{Questions}
\label{\detokenize{tutorials/work/analyse/index:questions}}\label{\detokenize{tutorials/work/analyse/index::doc}}\begin{itemize}
\item {} 
« Des outils en fabrication numériques seront prêtés » et « un accompagnement sera réalisé » (page 28). Une compétence pédagogique secondaire qui devra être détenue par ceux qui animeront les cours.

\end{itemize}

On m’a dit que parfois je formerais les bibliothécaires, mais que j’animerais tout les cours, mais en même temps on m’a parlé de prêts, pour moi aussi ce n’est pas clair. Et donc en même temps je ne vois pas trop l’intérêt de former les bibliothécaires s’il n’y a que moi qui fait les ateliers, à part peut être pour leur expliquer l’intérêt de la fabrication numérique et pourquoi en faire dans leur bibliothèque.
\begin{itemize}
\item {} 
« Vous » dites que les activités proposées de manière proactive seraient en général bien reçues.

\end{itemize}

Il y a beaucoup de petites structures, bénévoles, personnes âgées. Le chauffeur qui distribue les livres a parlé de la fabrication numérique, les personnes n’étaient pas enthousiastes pour des ateliers ou veulent faire des événements accès sur le livre comme la lecture d’un conte.

\begin{sphinxadmonition}{note}{Note:}
Il y a une contrainte qui n’a pas été évoquée c’est la nuisance sonore trop importante pour une bibliothèque pendant le déroulement des ateliers entre les participants et le fonctionnement des machines.
\end{sphinxadmonition}

Le mieux ce sera de le faire dans les médiathèques qui proposent une salle à part insonorisée. Il faudra voir pour le faire peut\sphinxhyphen{}être en heure de fermeture au public, mais il y aura moins de monde, ce sera compliqué pour cibler des enfants sauf pendant les vacances. Parfois la structure n’est pas accessible en dehors des quelques heures d’ouvertures au public quand c’est géré par des bénévoles.
\begin{itemize}
\item {} 
Vous soulevez le problème des utilisateurs qui veulent leur création rapidement et que 20 minutes c’est trop long. Pour l’impression 3D, il faudra attendre plusieures heures voire demander de repasser le lendemain. Ca peut être compliqué si l’atelier et déjà fait un week end et les médiathèques géré par des bénévoles avec peu d’horaires d’ouverture.

\end{itemize}


\section{Bases du fonctionnement du Lab itinérant}
\label{\detokenize{tutorials/work/analyse/index:bases-du-fonctionnement-du-lab-itinerant}}
(Résumé et suggestions de Bekhoucha Danyl à partir des études de Plascilab et de Chronos)


\section{L’intérêt de la fabrication numérique}
\label{\detokenize{tutorials/work/analyse/index:l-interet-de-la-fabrication-numerique}}
\begin{sphinxadmonition}{important}{Important:}
Témoignage important de la Médiathèque des Ulis : « La fabrication numérique ça n’est déjà plus très nouveau pour le grand public. À la médiathèque, notre rôle d’initiation et de découverte était un peu dépassé. C’est pourquoi nous nous sommes recentrés sur l’initiation au code et à la programmation, ainsi que sur la robotique. On ne veut pas devenir le service après\sphinxhyphen{}vente de Darty. On montre, mais on ne fait pas à la place de ».
\end{sphinxadmonition}

Lors des initiations de découverte, il sera important de commencer par parler de l’intérêt, car beaucoup voient la fabrication numérique comme un « gadget pour les geeks ».
Des exemples de gros projets pourront être montrés comme le prototypage, la fabrication de maisons, voitures, d’avions, robots, appareils médicaux.
Les progrès et combinaisons de la fabrication numérique combinés à d’autres secteurs du numérique comme l’intelligence artificielle donneront naissance à l’industrie 4.0 ou 4e révolution industrielle.

S’initier à la fabrication numérique permet d’acquérir des compétences à mettre en valeur:
\begin{itemize}
\item {} 
Renforcer ses compétences techniques et créatives pouvant être valorisées sur le marché du travail

\item {} 
Découvrir l’apprentissage en autodidacte, en coopération, le télétravail

\item {} 
Découvrir la culture « Libre », le partage de ses créations et documentations afin de permettre aux autres de reproduire et de contribuer pour améliorer

\item {} 
Croiser différentes technologies et cultures: impression 3D, logiciels de modélisation 3D, programmation, robotique, création vidéo, etc.

\end{itemize}


\section{Objectifs des ateliers}
\label{\detokenize{tutorials/work/analyse/index:objectifs-des-ateliers}}
\begin{sphinxadmonition}{note}{Note:}\begin{itemize}
\item {} 
Les cours seront rédigés dans l’esprit de la philosophie des « Makers ». Facilement accessibles et sous une licence permissive Creative Commons afin de faire profiter la communauté

\item {} 
Globalement la médiation à la fabrication numérique aura une dimension conviviale axée sur la découverte

\end{itemize}
\end{sphinxadmonition}

Les ateliers devront satisfaire des attentes multiples:
\begin{itemize}
\item {} 
Pour les jeunes des projets libres, créatifs, ludiques et innovants

\item {} 
Pour la population âgée moins à l’aise avec les outils numériques des projets guidés

\end{itemize}

\begin{sphinxadmonition}{note}{Note:}
Les médiathèques font plus facilement le lien avec l’art et la création numérique alors que les Fablabs sont plus orientés sur les sciences et la réparation.
\end{sphinxadmonition}

Trouver des intérêts:
\begin{itemize}
\item {} 
En faisant un lien entre le numérique et la vie quotidienne en faisant des objets directement utiles

\item {} 
Interroger les publics sur leurs intérêts afin de baser les ateliers dessus

\end{itemize}


\section{Mise en avant des actions}
\label{\detokenize{tutorials/work/analyse/index:mise-en-avant-des-actions}}\begin{itemize}
\item {} 
Le site de la MDE et Bibliosud serviront à promouvoir les actions

\item {} 
La réalisation de démonstrations en bibliothèque permettra de convaincre à la réalisation d’actions
\begin{itemize}
\item {} 
Un des arguments des actions sera aussi de promouvoir les lieux pour attirer des usagers

\end{itemize}

\item {} 
Le site \sphinxurl{https://www.tierslieuxedu.org/} permet de rentrer en contact avec les Makers et associations, d’accéder à des ressources, etc.

\end{itemize}


\section{Lieux potentiels}
\label{\detokenize{tutorials/work/analyse/index:lieux-potentiels}}\begin{itemize}
\item {} 
Les tiers lieux fixes et associations sont des lieux intéressants pour réaliser des activités et partenariats potentiels

\item {} 
Lieux collectifs: Associations, médiathèques, salles des fêtes, collèges ruraux (pour ce dernier certains participants ne seront pas volontaires, les ateliers seront plus difficiles)

\item {} 
Lieux non institutionnels: marchés, galeries marchandes et équipement patrimoniaux

\end{itemize}

\begin{sphinxadmonition}{note}{Note:}
Il y a par exemple le château de Méréville, le domaine de Chamarande et Montauger.
\end{sphinxadmonition}
\begin{itemize}
\item {} 
Les lieux devront être correctement répartis dans le territoire du sud de l’Essonne

\end{itemize}

\begin{sphinxadmonition}{note}{Note:}
La fabrication numérique sera l’occasion de faire découvrir d’autres lieux notamment les Fablabs au nord de l’Essonne pour ceux souhaitant continuer et accéder à des outils plus poussés.
\end{sphinxadmonition}


\section{Conditions de réalisation des actions}
\label{\detokenize{tutorials/work/analyse/index:conditions-de-realisation-des-actions}}
\begin{sphinxadmonition}{note}{Note:}
Les activités ne seront pas portées par une seule personne, mais par une équipe avec des compétences complémentaires.
\end{sphinxadmonition}
\begin{itemize}
\item {} 
Les activités seront toutes gratuites dans une logique de service public

\item {} 
Il faudra faire au minimum à la place du participant, afin de refuser une logique « servicielle »

\item {} 
Les mercredis après\sphinxhyphen{}midi et vacances scolaires seront à privilégier

\item {} 
Les amplitudes horaires élargies seront privilégiées en proposant ponctuellement des activités le soir et le weekend

\end{itemize}

\begin{sphinxadmonition}{attention}{Attention:}
Le faible nombre de transports en commun en zone rurale le soir et weekend seront un inconvénient en particulier pour le retour le soir.
\end{sphinxadmonition}
\begin{itemize}
\item {} 
Il est recommandé de ne pas excéder plus de 10 personnes aux ateliers afin de simplifier leurs déroulements

\item {} 
Les ateliers plus complexes peuvent se faire avec un nombre de participants restreint

\item {} 
Il peut y avoir un « assistant » pour aider lors des ateliers plus complexes

\item {} 
Les ateliers ne doivent pas être trop long, deux heures seraient le maximum et 1h 30 en moyenne

\end{itemize}


\section{S’adapter au public}
\label{\detokenize{tutorials/work/analyse/index:s-adapter-au-public}}\begin{itemize}
\item {} 
Faire un lien entre la fabrication numérique et les jeux vidéos ou la robotique par exemple afin de leur donner envie de participer

\item {} 
Généralement les garçons sont plus intéressés par la fabrication numérique que les filles

\item {} 
Des projets en équipe collectifs pourront être mis en place

\item {} 
Pour les enfants, il faut une approche plus ludique et intuitive, basée sur la créativité avec un résultat rapide et peu de mots techniques

\item {} 
Pour les personnes âgées, cela peut se faire sur plusieurs séances en réalisant des objets du quotidien. La convivialité sera mise en avant (pause quand ils le souhaitent par exemple)

\end{itemize}

\begin{sphinxadmonition}{note}{Note:}\begin{itemize}
\item {} 
Il faut réaliser des actions avec des personnes volontaires plutôt qu’avec un public scolaire auquel on impose la présence

\item {} 
Il y a un vieillissement de la population, des départs massifs à la retraite et une baisse marquée des moins de 20 ans

\end{itemize}
\end{sphinxadmonition}


\section{Les contraintes des lieux}
\label{\detokenize{tutorials/work/analyse/index:les-contraintes-des-lieux}}
\begin{sphinxadmonition}{important}{Important:}
Globalement la fabrication numérique ne bénéficie plus de l’effet nouveauté d’il y a quelques années.
\end{sphinxadmonition}
\begin{itemize}
\item {} 
Il y a peu d’équipement dans de nombreuses zones rurales

\item {} 
Il y a un problème d’espace disponible pour mener des ateliers

\item {} 
Peu de compétences et connaissances des possibilités de la fabrication numérique

\item {} 
Les machines et les regroupements génèrent du bruit, des espaces fermés seront souvent nécessaires

\end{itemize}


\section{Contraintes techniques}
\label{\detokenize{tutorials/work/analyse/index:contraintes-techniques}}\begin{itemize}
\item {} 
La fabrication des objets en particulier avec l’imprimante 3D peut être très longue et nécessiter au participant de revenir plusieurs jours après pour récupérer sa création ce peut être compliqué pour le lab itinérant et pour le participant.

\item {} 
Les participants veulent repartir avec quelque chose tout de suite.

\end{itemize}

\begin{sphinxadmonition}{note}{Note:}
Il faudra plusieurs imprimantes 3D pour imprimer plus rapidement les créations.
\end{sphinxadmonition}

\begin{sphinxadmonition}{important}{Important:}
Les dépenses de fonctionnement sous souvent sous\sphinxhyphen{}estimés:
\sphinxhyphen{} L’entretien et la maintenance des machines
\sphinxhyphen{} le renouvèlement des machines
\sphinxhyphen{} l’achat de matériaux et de consommables
\end{sphinxadmonition}
\begin{itemize}
\item {} 
Il faut prévoir un temps de préparation des ateliers
\begin{itemize}
\item {} 
Des temps de formation des intervenants

\item {} 
La rédaction

\item {} 
L’acheminement et mise en place des machines

\item {} 
Le réglage et l’entretien des machines en dehors des ateliers

\end{itemize}

\end{itemize}

\begin{sphinxadmonition}{note}{Note:}\begin{itemize}
\item {} 
Avoir des compétences relatives au réglage et maintenance basique plutôt que de passer par un prestataire afin de limiter les dépenses, sauf si cela annule la garantie

\item {} 
Le filament biodégradable et recyclable PLA sera utilisé afin d’être davantage écoresponsable

\end{itemize}
\end{sphinxadmonition}


\section{Les conclusions données dans l’étude}
\label{\detokenize{tutorials/work/analyse/index:les-conclusions-donnees-dans-letude}}
Les compétences attendues du médiateur en fabrication numérique sont:
\begin{itemize}
\item {} 
Une compétence technique assez forte sur le numérique, en envisageant une montée en compétence progressive sur les questions d’éducation, de pédagogie, etc. qui apparait peut être comme une compétence secondaire devant être détenue par ceux qui animeront les activités sur le terrain

\item {} 
Une capacité à « activer » des ressources ou des compétences externes, en travaillant par exemple en collaboration avec Canopé (qui propose déjà de la formation de formateurs), ou avec des Fablabs existants (pour accéder ponctuellement à des machines de pointe par exemple)

\end{itemize}


\chapter{Réunion 19/06}
\label{\detokenize{tutorials/work/reunion/index:reunion-19-06}}\label{\detokenize{tutorials/work/reunion/index::doc}}
\sphinxstylestrong{Date:} 19/06/2020

\sphinxstylestrong{Participants:} Aurélia, Kevin, Danyl, Maud


\section{Kevin}
\label{\detokenize{tutorials/work/reunion/index:kevin}}

\subsection{Fait:}
\label{\detokenize{tutorials/work/reunion/index:fait}}\begin{itemize}
\item {} 
La borne interactive part à Forges\sphinxhyphen{}Les\sphinxhyphen{}Bains fin juillet

\item {} 
Appel à projets Oculus vient d’être lancé pour 5 bibliothèques
\begin{itemize}
\item {} 
Dépouillement en aout

\item {} 
Déploiement en octobre

\end{itemize}

\item {} 
Argumentaire Wifi réalisé

\end{itemize}


\subsection{À faire:}
\label{\detokenize{tutorials/work/reunion/index:a-faire}}\begin{itemize}
\item {} 
S’occuper du Cartomaton: fiche atelier, outil, catalogage

\item {} 
Travailler sur Les Petits Loups, mettre fiches ateliers

\item {} 
Mise au point Bibliosud avec Nathalie Dauvilliers

\item {} 
Communication bornes, Switch, etc

\end{itemize}


\subsection{Informations:}
\label{\detokenize{tutorials/work/reunion/index:informations}}\begin{itemize}
\item {} 
En congé fin juillet

\item {} 
32\% d’inscrits en plus pendant le confinement sur bibliocollege

\end{itemize}


\section{Danyl}
\label{\detokenize{tutorials/work/reunion/index:danyl}}

\subsection{Fait:}
\label{\detokenize{tutorials/work/reunion/index:id1}}\begin{itemize}
\item {} 
Prise en main des machines, plus que les fonctions avancées à faire et la Bernina (brodeuse) à apprendre

\item {} 
Documentation de l’utilisation des machines

\end{itemize}


\subsection{Informations:}
\label{\detokenize{tutorials/work/reunion/index:id2}}\begin{itemize}
\item {} 
Site MDE:
\begin{itemize}
\item {} 
L’onglet Numerique devient Innovation

\item {} 
Mettre en ligne des fiches action s’inspirant du PDDLP
\begin{itemize}
\item {} 
Fiche projet (grands projets globaux) et fiche outils (mode d’emploi des outils éventuellement en prêt)

\end{itemize}

\item {} 
Fiche atelier

\end{itemize}

\end{itemize}

\begin{sphinxadmonition}{note}{Note:}
Il faudra une charte graphique commune pour les fiches.
\end{sphinxadmonition}
\begin{itemize}
\item {} 
Le médiateur de Sainte\sphinxhyphen{}Geneviève\sphinxhyphen{}des\sphinxhyphen{}Bois viendra le jeudi 27 aout

\item {} 
Journée de formation à Draveil à Culture Numérique le 7 et 8 septembre sur la fabrication numérique pour les agents de la MDE

\item {} 
25 septembre atelier avec les partenaires à Chamarande
\begin{itemize}
\item {} 
Puis restitutions des ateliers et rédactions d’un catalogue d’offres envoyés au Cotech

\item {} 
Fin novembre présentation officielle aux élus

\end{itemize}

\item {} 
Onglet numérique sur le site de la MDE. Fiches actions \textgreater{} Fiches projet \textgreater{} Fiche outils qui décrivent les outils en prêt \textgreater{} Fiches ateliers

\end{itemize}


\section{Maud}
\label{\detokenize{tutorials/work/reunion/index:maud}}

\subsection{Fait:}
\label{\detokenize{tutorials/work/reunion/index:id3}}\begin{itemize}
\item {} 
Réalisation d’un bilan pour les CDI et expérimentateurs

\item {} 
Propositions de fiche d’idées créatives à destrination des professeurs de technologie et d’art plastique du sud du département avec les outils créamix: touchboard

\item {} 
Partenariat avec Canopé, le CD, webinar autour du numérique pour enfants

\item {} 
Réunion Copil pour expliquer bibliocollege pendant le confiement

\item {} 
Fonctionnalité prof express activé (tableau blanc numérique en ligne) le jeudi et vendredi de 17h à 20h

\end{itemize}


\subsection{Informations:}
\label{\detokenize{tutorials/work/reunion/index:id4}}\begin{itemize}
\item {} 
Bibliocollège compte +21\% d’inscrits depuis le confinement

\end{itemize}


\section{Transversalité}
\label{\detokenize{tutorials/work/reunion/index:transversalite}}\begin{itemize}
\item {} 
Formation en septembre sur le portail pour répondre aux questions et effectuer une aide de niveau 2 (après que l’utilisateur ai lu les informations du site)

\item {} 
formation jeux vidéo octobre 2020 et en 2021

\item {} 
Inscription à 2021 à Plascilab, cela permettra d’aller en formation

\item {} 
Attribuer un Kakémono pour l’Oculus Rift et les machines du Lab

\item {} 
Formation par des prestataires sur l’initiation à la fabrication numérique le 2 décembre et mi\sphinxhyphen{}décembre

\item {} 
Cartes Steam bientôt disponible

\end{itemize}


\chapter{Reunion 2}
\label{\detokenize{tutorials/work/reunion/index:reunion-2}}
PLACEHOLDER


\chapter{Synthèse: Espaces de création numérique en bibliothèque}
\label{\detokenize{tutorials/work/livre/index:synthese-espaces-de-creation-numerique-en-bibliotheque}}\label{\detokenize{tutorials/work/livre/index::doc}}

\section{Projets}
\label{\detokenize{tutorials/work/livre/index:projets}}\begin{itemize}
\item {} 
Console de jeu \sphinxurl{https://cyrzbib.net/2017/02/04/operation-jaja-box-1ere-partie/}

\item {} 
Blog intéressant avec des vidéos: \sphinxurl{https://cyrzbib.net/2020/05/11/le-corolab-de-la-mediatheque/} (+ vidéos incluses)

\item {} 
Wheeldo fablab mobile (n’existe plus): \sphinxurl{http://wheeldo.eu/}

\end{itemize}


\section{Lexique}
\label{\detokenize{tutorials/work/livre/index:lexique}}\begin{itemize}
\item {} 
Makerspace: Un makerspace est un tiers lieu de type atelier de fabrication numérique, évolution du hackerspace, ouvert au public et mettant à disposition des machines\sphinxhyphen{}outils et machines\sphinxhyphen{}outils à commande numérique habituellement réservées à des professionnels dans un but de prototypage rapide ou de production à petite échelle. \sphinxurl{https://fr.wikipedia.org/wiki/Makerspace}

\item {} 
Hackerspace/Hacklab: Un hackerspace ou Hacker house est un tiers\sphinxhyphen{}lieu où des gens avec un intérêt commun (souvent autour de l’informatique, de la technologie, des sciences, des arts…) peuvent se rencontrer et collaborer. Les Hackerspaces peuvent être vus comme des laboratoires communautaires ouverts où des gens (les hackers) peuvent partager ressources et savoir. \sphinxurl{https://fr.wikipedia.org/wiki/Hackerspace}

\item {} 
Tiers lieu: cadre public informel, qui permet de créer une communauté vivante, qui favoris un sentiment d’appartenance.

\item {} 
Fablab (Fabrication Laboratory): lieu ouvert au public, met à disposition des outils notamment des machines\sphinxhyphen{}outils pilotées par ordinateur pour la conception et la réalisation d’objets.

\item {} 
Biohackerspace: La biologie participative désigne une approche de la biologie contributive, indépendante ou collaborative avec des laboratoires académiques ou industriels. Il s’agit d’individus (néophytes, amateurs, ou expérimentés), d’associations ou de petites entreprises dont les visées sont souvent non lucratives, dans une démarche de science ouverte ou éducative, ou lucratives. \sphinxurl{https://fr.wikipedia.org/wiki/Biologie\_participative}

\item {} 
Livinglab: Le living lab est une méthodologie où citoyens, habitants, usagers sont considérés comme des acteurs clés des processus de recherche et d’innovation. \sphinxurl{https://fr.wikipedia.org/wiki/Living\_lab}

\item {} 
Learning lab: Les learning labs sont des lieux d’expérimentation en innovation pédagogique. Souvent soutenus par des entreprises ou regroupement d’universités et centres de formation, leur développement s’inscrit dans le mouvement des moocs, de l’intelligence collective et du web 2.0. \sphinxurl{https://fr.wikipedia.org/wiki/Learning\_lab}

\item {} 
Scrapbooking: Le scrapbooking, ou créacollage, collimage (francisations principalement utilisées au Québec), est une forme de loisir créatif consistant à introduire des photographies dans un décor en rapport avec le thème abordé, dans le but de les mettre en valeur par une présentation plus originale qu’un simple album photo. \sphinxurl{https://fr.wikipedia.org/wiki/Scrapbooking}

\item {} 
do•doc: Conçu pour documenter et créer des récits à partir d’activités pratiques. \sphinxurl{https://latelier-des-chercheurs.fr/outils/dodoc}

\item {} 
ECN: Espace de Création Numérique

\item {} 
Fork: un fork (terme anglais signifiant « fourche », « bifurcation », « embranchement ») est un nouveau logiciel créé à partir du code source d’un logiciel existant lorsque les droits accordés par les auteurs le permettent: ils doivent autoriser l’utilisation, la modification et la redistribution du code source. C’est pour cette raison que les forks se produisent facilement dans le domaine des logiciels libres.

\item {} 
Littératie numérique: La littératie numérique résulte de la juxtaposition des termes « littératie » et « numérique ». Elle se définit en deux temps. La littératie, est définie par l’OCDE comme étant « l’aptitude à comprendre et à utiliser l’information écrite dans la vie courante, à la maison, au travail et dans la collectivité en vue d’atteindre des buts personnels et d’étendre ses connaissances et ses capacités »1. Le terme « numérique » est un terme polysémique. Il recouvre plusieurs notions : l’informatique, la technologie, l’information, le visuel et la communication. La littératie numérique s’apprécie comme la capacité d’un individu à participer à une société qui utilise les technologies de communication numériques dans tous ses domaines d’activité. \sphinxurl{https://fr.wikipedia.org/wiki/Litt\%C3\%A9ratie\_num\%C3\%A9rique}

\item {} 
Design Thinking: Le design thinking (littéralement « penser le design »), en français démarche design ou conception créative, est une méthode de gestion de l’innovation. Il se veut une synthèse entre pensée analytique et pensée intuitive. Il fait partie d’une démarche globale appelée design collaboratif. Il s’appuie en grande partie sur un processus de co\sphinxhyphen{}créativité impliquant des retours de l’utilisateur final. Contrairement à la pensée analytique, le design thinking est un ensemble d’espaces qui s’entrecroisent plutôt qu’un processus linéaire ayant un début et une fin. \sphinxurl{https://fr.wikipedia.org/wiki/Design\_thinking}

\end{itemize}

Pour être fablab la charte du MIT doit être respectée.
\begin{itemize}
\item {} 
Espace HOMAGO: HO (Hang\sphinxhyphen{}Out), MA (Mess Arround), GO (Geek Out)

\item {} 
Repair Café/Repair Goûté: Un repair café (littéralement café de réparation) est un atelier consacré à la réparation d’objets et organisé à un niveau local sous forme de tiers lieu, entre des personnes qui habitent ou fréquentent un même endroit (un quartier ou un village, par exemple). \sphinxurl{https://fr.wikipedia.org/wiki/Repair\_Caf\%C3\%A9}

\item {} 
DIY: Do It Yourself (faire par soi\sphinxhyphen{}même)

\item {} 
DIWO: Do It With Other

\item {} 
Learning\sphinxhyphen{}by\sphinxhyphen{}doing: Apprendre en faisant

\item {} 
Lifelong Learning: l’éducation tout au long de la vie

\end{itemize}


\bigskip\hrule\bigskip



\section{Préface}
\label{\detokenize{tutorials/work/livre/index:preface}}

\subsection{Fablabs et innovation sociale : de quoi parle\sphinxhyphen{}t\sphinxhyphen{}on ? (p. 13)}
\label{\detokenize{tutorials/work/livre/index:fablabs-et-innovation-sociale-de-quoi-parle-t-on-p-13}}\begin{itemize}
\item {} 
l’initiative des Fablabs vient de Neil Gershenfeld un professeur américain du MIT

\item {} 
la France est leader Européen avec environ 500 fablabs

\item {} 
le but: la démocratisation des savoirs numériques et l’apprentissage par projet

\item {} 
le fonctionnement: horizontal et ouverts

\item {} 
partage de savoirs gratuits

\item {} 
chacun reste propriétaire de ce qu’il conçoit, mais documente son expérience sur internet

\end{itemize}

\sphinxstylestrong{Valeurs:}
\begin{itemize}
\item {} 
individualisme

\item {} 
participation directe

\item {} 
bienveillance

\item {} 
partage gratuit

\item {} 
pouvoir distribué

\item {} 
relations horizontales

\end{itemize}


\bigskip\hrule\bigskip



\section{Introduction}
\label{\detokenize{tutorials/work/livre/index:introduction}}

\subsection{Espace de Création Numérique (ECN) en médiathèque: devenir bibliomaker ? (p. 19)}
\label{\detokenize{tutorials/work/livre/index:espace-de-creation-numerique-ecn-en-mediatheque-devenir-bibliomaker-p-19}}
Le mouvement maker c’est aussi l’apprentissage convivial.

Les valeurs sont communes avec les médiathèques:
\begin{itemize}
\item {} 
accès au savoir

\end{itemize}
\begin{itemize}
\item {} 
partage de connaissances

\item {} 
développement de l’imagination et créativité

\end{itemize}


\bigskip\hrule\bigskip



\section{S’inspirer : que nous apprennent les makers ?}
\label{\detokenize{tutorials/work/livre/index:s-inspirer-que-nous-apprennent-les-makers}}

\subsection{De quels lieux parle\sphinxhyphen{}t\sphinxhyphen{}on ? (p. 25)}
\label{\detokenize{tutorials/work/livre/index:de-quels-lieux-parle-t-on-p-25}}
Voir les définitions.

Les caractéristiques d’un Fablab c’est:
\begin{itemize}
\item {} 
le plaisir, le jeu, la passion

\item {} 
l’engagement

\item {} 
la coopération directe sans hiérarchie

\item {} 
pas de profit (mouvement du logiciel Libre)

\item {} 
la créativité

\item {} 
la socialisation

\item {} 
apprentissage informel

\item {} 
expérimentation, innovation, pratique, autonomie

\end{itemize}


\subsection{SimplonLab, le fablab social (p.29)}
\label{\detokenize{tutorials/work/livre/index:simplonlab-le-fablab-social-p-29}}
Les activités:
\begin{itemize}
\item {} 
textile et graphisme: brodeuse numérique, machine à coudre, plotter vinyle, sérigraphie

\item {} 
menuiserie: outillage à bois type perceuses, visseuses, scies portatives

\item {} 
électronique: microcontrôleurs, microprocesseurs, composants

\item {} 
machines: découpeuse laser, imprimantes 3D, postes soudure, espace d’exposition, entrepôt des projets et postes de travail sur les logiciels

\end{itemize}

Techniques pour identifier les partenaires:
\begin{itemize}
\item {} 
identifier les besoins des publics

\item {} 
construire des partenariats

\item {} 
faire découvrir les fablabs aux personnes de quartier

\end{itemize}

Au lancement du projet fablab:
\begin{itemize}
\item {} 
identifier les associations

\item {} 
les contacter par mail

\item {} 
se rendre dans leurs locaux

\item {} 
prendre le temps de présenter notre action

\end{itemize}

Puis:
\begin{itemize}
\item {} 
Participation à des rencontres collectives

\end{itemize}

\begin{sphinxadmonition}{important}{Important:}
Il est important de construire des relations qualitatives et durables avec un petit nombre de structures
\end{sphinxadmonition}


\subsection{Pot au fab : un fablab solidaire (p. 35)}
\label{\detokenize{tutorials/work/livre/index:pot-au-fab-un-fablab-solidaire-p-35}}\begin{itemize}
\item {} 
mêle la fabrication numérique avec des savoir\sphinxhyphen{}faire plus traditionnels comme la cuisine afin de favoriser rencontre et convivialité et d’offrir un point d’entrée bien identifiable pour les futurs stagiaires

\end{itemize}


\subsection{Encourager l’inclusion du public féminin en fablabs (p. 39)}
\label{\detokenize{tutorials/work/livre/index:encourager-l-inclusion-du-public-feminin-en-fablabs-p-39}}\begin{itemize}
\item {} 
« Les fablabs se donnent pour objectif d’être des lieux ouverts à tous, où chacun peut venir apprendre et partager ses connaissances dans une ambiance conviviale ».

\item {} 
« {[}…{]} favoriser l’égalité et la cohésion sociale »

\item {} 
Les fablabs donnent un accès gratuit à des machines de haute technologie proposées à un coût élevé dans le commerce, ont pour but de procéder à l’empowerment du grand public ».

\item {} 
Les fablabs remettent en question la notion de hiérarchie {[}…{]}, tout le monde, même un débutant est capable de proposer des idées et connaissances » = atmosphère de solidarité

\item {} 
« L »univers des makers, du moins en Europe, est encore largement dominé par une population masculine » 15\% femmes et 85\% d’hommes

\item {} 
Plus les propositions d’ateliers seront diverses, plus le public le sera aussi.

\item {} 
Une fois sur place ils et elles ont pu s’initier à d’autres activités.

\item {} 
On peut demander au public ce qu’ils aimeraient voir et faire avec un sondage en ligne.

\end{itemize}


\subsection{Le fablab de la Cité des Sciences  et de l’Industrie et son wiki (p. 45)}
\label{\detokenize{tutorials/work/livre/index:le-fablab-de-la-cite-des-sciences-et-de-l-industrie-et-son-wiki-p-45}}\begin{itemize}
\item {} 
utilisation de docuwiki pour documenter les projets (do\sphinxhyphen{}ocratie)

\item {} 
l’une des tâches des médiateurs et médiatrices est l’intermédiation pour faciliter l’apprentissage par les pairs

\item {} 
préconise l’usage du Creative Commons, l’auteur\sphinxhyphen{}e reste auteur\sphinxhyphen{}e de plein droit

\end{itemize}


\subsection{From bits to atoms (p. 51)}
\label{\detokenize{tutorials/work/livre/index:from-bits-to-atoms-p-51}}
\begin{sphinxadmonition}{note}{Note:}
Pas d’information à retenir.
\end{sphinxadmonition}


\bigskip\hrule\bigskip



\section{Fabriquer à la bibliothèque}
\label{\detokenize{tutorials/work/livre/index:fabriquer-a-la-bibliotheque}}

\subsection{Pourquoi installer un Espace de Création Numérique à la bibliothèque ? (p. 59)}
\label{\detokenize{tutorials/work/livre/index:pourquoi-installer-un-espace-de-creation-numerique-a-la-bibliotheque-p-59}}\begin{itemize}
\item {} 
« Lorsque le territoire est déjà doté d’un vrai fablab, il ne s’agit pas d’offrir des services redondants, mais surtout d’être complémentaire.

\item {} 
« Bad Librariebuild collections. Good librarie build services. Great libraries build Communities » \sphinxhyphen{} David Lankes

\item {} 
Les mauvaises bibliothèques construisent des collections, les bonnes bibliothèques construisent des services, les super bibliothèques construisent des communautés. (page 62)

\item {} 
Certains dispositifs ponctuels permettent de faire découvrir au plus grand nombre ce qui se joue dans les fablabs à travers un événement festif et ludique après la découverte et l’initiation, tout semble en place pour que dans un 3e temps les usagers puissent développer leurs propres projets {[}…{]}.

\item {} 
Il sera temps de passer la main au fablab de quartier ou de la ville.

\end{itemize}

Rôle des ECN:
\begin{itemize}
\item {} 
Accès à l’information, à la formation, à l’éducation et à la culture

\item {} 
autoformation tout au long de la vie

\item {} 
vulgarisation de la culture scientifique

\end{itemize}

On apprend mieux quand:
\begin{itemize}
\item {} 
le contenu est au coeur de nos centres d’intérêt

\item {} 
nous interagissons avec nos pairs

\item {} 
nous nous donnons le droit de nous tromper

\item {} 
quand le lieu est convivial

\end{itemize}


\subsection{Des CDI aux tiers lieux : l’évolution des fonctions (p. 67)}
\label{\detokenize{tutorials/work/livre/index:des-cdi-aux-tiers-lieux-l-evolution-des-fonctions-p-67}}\begin{itemize}
\item {} 
Premier lieu: chez soi

\item {} 
Second lieu: travail

\item {} 
Troisième lieu: un lieu qui permet aux habitants d’une collectivité de se réunir de façon conviviale

\item {} 
Tiers lieu: ajoute une dimension de création, un processus de conception partagé

\end{itemize}

Une pratique clé identifiée dans les tiers lieux : celle de la documentation. Documenter pour faire un patrimoine d’informations en commun qui permette à la population qui ne peut pas venir dans le tiers lieu de se réapproprier ce qui y est conçu.

L’esprit de collaboration et l’horizontalité y sont maîtres, les élèves qui maîtrisent une technique sont encouragés à la transmettre à leurs camarades et aux professeurs, ce qui redonne confiance aux élèves en difficulté.

Les élèves n’ont souvent, pas envie de passer de la manipulation à l’écrit et de se mettre à la place des autres utilisateurs ayant besoin d’information. L’éducation des élèves et usagers à une culture et une philosophie maker est aussi destinée à leur faire percevoir l’intérêt de la constitution d’un fonds commun, et donc à motiver leurs pratiques de documentation.


\subsection{Des lieux de création à l’université ? (p. 75)}
\label{\detokenize{tutorials/work/livre/index:des-lieux-de-creation-a-l-universite-p-75}}
Le gamelab à l’université de Paris 5 est un laboratoire ouvert étudiant les usages ludiques et concevant à la fois des serious games, des escape games ou des jeux de plateau.


\subsection{BibLab : promouvoir la création numérique en ruralité (p. 81)}
\label{\detokenize{tutorials/work/livre/index:biblab-promouvoir-la-creation-numerique-en-ruralite-p-81}}
Depuis 2016 la Direction de la Lecture Publique (DLP) du Loir\sphinxhyphen{}et\sphinxhyphen{}Cher organise le \sphinxhref{http://lecture41.culture41.fr/bib-41/festival-numerique-vagabondag-e-s/1104-vagabondag-e-s-edition-2019}{Festival numérique Vagabondag(e)s}.

Edition 2019: « Dix\sphinxhyphen{}sept bibliothèques accueilleront les animations mises en place par la Direction de la lecture publique : Journée de découverte des robots, projet d’écriture collective, mapping vidéo, principes du cinéma d’animation, ateliers GIF animés, découverte des FabLab du département, jeux vidéo, photomontages ou éducation à l’image, etc. Ces activités permettront d’accompagner le public dans la découverte et l’utilisation des outils numériques novateurs. »

BibLab est un fablab itinérant et a pour objectif de permettre la mise en place d’ateliers d’initiation et de découverte dans l’ensemble du réseau départemental, quelle que soit la taille de la bibliothèque, la valorisation et la transmission de la culture scientifique et technique encore trop confidentielle dans nos établissements.
Cela permet aux bibliothèques qui le souhaitent de tester ce matériel, de voir les possibilités d’animations avec les publics et de s’équiper ou de nouer des partenariats avec l’un des fablab du département. BibLab est donc prêté sur projet et pour un temps suffisamment long pour permettre l’expérimentation (plusieurs semaines à plusieurs mois selon les projets).

Pour être facilement transportable et installable tout en étant ludique et visible dans les espaces, la DLP a fait fabriquer un flightcase qui se déploie comme un bureau: \sphinxhref{http://icietlab.cc/}{ici et lab}.

BibLab se compose de trois univers thématiques :
\begin{itemize}
\item {} 
l’univers makerspace propose des outils essentiels de création: imprimante 3D, stylos 3D, découpeuse papier et vinyle, cartes Makey Makey et Arduino, micro\sphinxhyphen{}ordinateurs Raspberry Pi, TouchPad, ordinateur DIY kano, Opad, Ipad Pro avec stylos optiques et enfin PC ous Windows et Linux.

\item {} 
Les contenus du kit Nos amis les robots sont axés sur les apprentissages ludiques autour du codage. Pour cela, la DLP s’équipe régulièrement de robots permettant cette initiation pour les tout\sphinxhyphen{}petits avec les robots Beebot et Cubetto, pour les enfants à partir de dix ans: robot Marty, pour tout public: le robot Cozmo entre jouet et robot programmable, ainsi qu’un drone programmable. Ces deux modules seront régulièrement enrichis de nouveaux matériels.

\item {} 
Éducation aux médias: contiens la table Mash\sphinxhyphen{}UP et ses accessoires.

\end{itemize}

Un dernier module complémentaire à BibLab autour du Nintendo Labo est prêté sous forme d’un kit clé en main comportant écran, switch et les différents Nintendo Labo, l’objectif étant de créer un lien entre jeu vidéo, codage et culture \sphinxstyleemphasis{Do It Yourself}.

Chaque module est prêté avec les fiches ateliers conçues par l’équipe de la DLP, ainsi qu’un cahier de retours d’expériences pour l’échange de bonnes idées entre emprunteurs et l’enrichissement des fiches d’ateliers proposées.

Le projet est transversal et non l’apanage des seuls bibliothécaires et animateurs, tout collègue intéressé peut s’investir. Pour cela un groupe de travail appelé, mission \sphinxstyleemphasis{Services innovants} a été mis en place. Ainsi des collègues au profil administratif ou technique font partie de l’équipe BibLab, tout comme deux collègues du réseau.
Ils proposent et élaborent des fiches ateliers, participent à la veille professionnelle, testent les machines ou ressources avant acquisition et animent des ateliers dans le réseau, en particulier lors du festival numérique.

Chaque module de BibLab a au minimum un binôme dédié, en particulier pour les formations et les réponses techniques, mais l’ensemble de l’équipe est en mesure d’animer des ateliers avec le matériel.

La DLP ne rencontrant qu’occasionnellement le public, l’aspect de formation et de transmission des connaissances et des compétences est essentiel pour qu’un projet, a fortiori expérimental, fonctionne auprès du réseau et trouve son public.
Ainsi, en plus des formations thématiques proposées dans son programme annuel, la DLP propose des rendez\sphinxhyphen{}vous de prise en main avec les équipes des bibliothèques emprunteuses. Venir dans les locaux permet également de caler l’installation de BibLab au mieux.

Chaque fois les ateliers sont préparés et animés en coordination avec les bibliothèques et avec les partenaires locaux tels que les maisons des jeunes, les centres sociaux, les associations et bien sûr les fablabs. Le tout dans une ambiance toujours conviviale autour par exemple d’un \sphinxstyleemphasis{petit\sphinxhyphen{}déjeuner numérique}.


\subsection{L’impression 3D à la médiathèque\sphinxhyphen{}ludothèque de Chassieu (p. 87)}
\label{\detokenize{tutorials/work/livre/index:l-impression-3d-a-la-mediatheque-ludotheque-de-chassieu-p-87}}
En novembre 2015, le projet d’acquisition d’une imprimante 3D voit le jour {[}…{]} L’achat doit en théorie permettre de proposer des animations autour de la modélisation et de l’impression 3D, mais elle est également considérée au quotidien comme un outil internet pour toute l’équipe : le remplacement des pièces de jeu abîmées de la ludothèque, l’impression de matériel pour les tablettes (fixation, pieds…) ou l’utilisation de l’impression 3D en support d’autres animations (impression de trophées).
Le chois se porte sur une imprimante fabriquée en France, qui bénéficie d’un bon service après\sphinxhyphen{}vente, et dont le coût est abordable (400 euros).
Une fois l’achat fait, il faudra encore attendre quelques mois avant que l’équipe ne soit capable de proposer un atelier, élaboré avec l’agent en service civique, qui assure également les ateliers informatiques auprès des usagers.

Au départ l’équipe n’était pas familière avec l’impression 3D ni avec la modélisation 3D.{[}…{]} La prise en main technique a donc essentiellement concerné les personnes en service civique et la coordinatrice numérique. Elle n’a pas été de tout repos, la machine achetée connaissant au début quelques bugs et autres défauts. {[}…{]} Nous nous sommes beaucoup appuyé\sphinxhyphen{}es sur la communauté existante sur internet, mais aussi sur les compétences et la patience de l’équipe.

il faut absolument avoir la motivation pour dépanner la machine. Nous avons parfois pu passer plusieurs heures sur une buse bouchée ou un plateau déréglé.
\begin{itemize}
\item {} 
\sphinxhref{https://mediathequemargueriteduras.wordpress.com/2017/08/09/journal-de-la-prise-en-main-de-limprimante-3d-dagoma-discoeasy/}{Le carnet de Marguerite (Journal de la prise en main de l’imprimante 3d Dagoma DiscoEasy)}

\end{itemize}

Le premier atelier: modélisation 3D et impression, sur une thématique choisie.
Les premiers ateliers étaient destinés à tous et toutes, mais les premier\sphinxhyphen{}ères inscrit\sphinxhyphen{}es étaient majoritairement des enfants. En pratique, leurs parents étaient extrêmement curieux du fonctionnement.

Cet engouement nous a poussés à proposer deux types d’ateliers : les ateliers modélisation impression, et les ateliers découverte.

Pour mener à bien les ateliers découverte, nous choisissions des modèles de moins de dix minutes à imprimer : nous avons également investi dans un stylo 3D, et nous nous sommes positionnés sur des matinées ou des après\sphinxhyphen{}midi entières, sans inscription, pour pouvoir répondre aux questions et satisfaire la curiosité des usager\sphinxhyphen{}ères petit\sphinxhyphen{}es et grand\sphinxhyphen{}es. D’où l’importance d’être deux pour l’encadrement.
Cette formule a l’avantage d’être rapide à mettre en place, facile à réaliser, mais surtout elle permet à toute personne curieuse, mais qui ne serait pas inscrite à un atelier long de s’arrêter, de poser des questions.

Dans le même temps, les imprimantes 3D destinées aux particuliers s’améliorent techniquement et simplifient au maximum leur utilisation. L’imprimante 3Den libre\sphinxhyphen{}service, déjà proposée dans certaines bibliothèques et plus largement dans les fablab, nous paraît donc abordable pour la médiathèque.
Proposer une imprimante qui ne soit pas intégrée à l’espace numérique, mais située à l’accueil, en libre\sphinxhyphen{}service, en développant des créneaux horaires où les usager\sphinxhyphen{}ères pourront imprimer. Toute l’équipe doit donc être en mesure de renseigner et d’accompagner le public dans la réalisation de ses impressions.
Nous avons libéré des créneaux de trois heures maximum, le mercredi et le samedi. Les créneaux sont accessibles sur inscriptions.
Nous fonctionnons , comme pour les impressions papier, avec une carte d’impression. Elle est facturée 2,50 euros les 50 grammes de fil.
L’objectif est de permettre à tous\sphinxhyphen{}tes de s’approprier ce nouveau service, en mettant l’accent sur le recyclage et la réparation d’objet, mais aussi sur la création.

Dès septembre 2019, nous proposerons régulièrement des ateliers de modélisation, des projets scolaires à envisager autour de création de jeu avec la ludothèque, un partenariat avec le centre de loisirs pendant les vacances scolaires, des projets communs à envisager en partenariat avec d’autres services de la mairie.


\subsection{Le collège, la médiathèque et le fablab (p. 93)}
\label{\detokenize{tutorials/work/livre/index:le-college-la-mediatheque-et-le-fablab-p-93}}
La question numérique est aujourd’hui au coeur de la vie des citoyens : généralisation des démarches administratives en ligne pour des structures socio\sphinxhyphen{}économiques (CAF, pôle\sphinxhyphen{}emploi, sécurité sociale), intégration dans les programmes scolaires de la programmation, du coding, du coding et de la robotique, multiplication des outils numériques comme les smartphones, tablettes ou outils robotisés (voitures, maisons, appareils connectés).

La bibliothèque trouve dans le makerspace une nouvelle expression de la diffusion des connaissances au sens large et de la culture scientifique en particulier. En créant un espace d’échange ouvert à tous et gratuit, la bibliothèque remplit sa mission d’ouverture tout en s’adaptant à son temps.

« Les outils numériques offrent cette opportunité d’intensifier le processus de co\sphinxhyphen{}construction, de partage et de diffusion des connaissances, en favorisant la participation de chacun ».

\sphinxstylestrong{Offres de services:}
\begin{itemize}
\item {} 
Atelier d’impression 3D \sphinxhyphen{} Conception et la création d’objets divers. Découverte du fonctionnement de l’imprimante 3D à partir de la création de bijoux grâce à un tutoriel de manipulation du logiciel Sketchup. Possibilité de proposer au public d’apporter des objets détériorés et de réfléchir aux diverses possibilités de réparation qu’offre l’imprimante 3D ;

\item {} 
Robotique et cartes électroniques programmables. Cet atelier est l’occasion de découvrir un domaine de plus en plus présent dans notre quotidien et d’initier le public à la programmation. Découverte d’un robot programmable (Lego Mindstorm et Lego WeDo 2.0) autour de réalisation d’activités ludiques. La programmation s’effectue via des tablettes. Découverte des possibilités offertes par les cartes électroniques programmables Arduino ;

\item {} 
Atelier de musique assistée par ordinateur (MAO). Une première approche de composition musicale via l’ordinateur et utilisation de claviers maîtres. La musique assistée par ordinateur consiste à concevoir une instrumentale, mais aussi à s’initier à la pratique instrumentale (guitare, piano, instruments électroniques) ;

\item {} 
Création de jeux vidéo. Utilisation du logiciel de programmation utilisé en grande partie dans l’éducation nationale (Scratch) et d’une plateforme de tutoriels (Google CS first). Des ateliers de création vidéo (dessins, interactivité, musique, bruitages, scénario, etc), en utilisant tous les logiciels, gratuits (Blender, Gimp, Audacityà ;

\item {} 
Découpe vinyle. Réalisation de productions plastiques grâce à la technique du Scrapbooking, création de magnets ou de tatouages.

\end{itemize}


\subsection{Le fablab de la médiathèque Brossard (Québec, CA) (p. 103)}
\label{\detokenize{tutorials/work/livre/index:le-fablab-de-la-mediatheque-brossard-quebec-ca-p-103}}
Il y a un immense potentiel de collaboration entre l’équipe fablab et médiathèque, mais l’absence d’objectifs communs explicités, une division des tâches stricte et la faiblesse des liens de communication déployés créent deux mondes quasi hermétiques au niveau organisationnel.

L’essence participative et collaborative des fablabs est parfois difficile à saisir pour le commun des mortels qui pense en termes de produits et de services.


\bigskip\hrule\bigskip



\section{Repenser notre posture professionnelle}
\label{\detokenize{tutorials/work/livre/index:repenser-notre-posture-professionnelle}}

\subsection{Et le bibliothécaire dans tout ça ? (p. 109)}
\label{\detokenize{tutorials/work/livre/index:et-le-bibliothecaire-dans-tout-ca-p-109}}
La question de la présence d’espaces créatifs numérique en bibliothèque change\sphinxhyphen{}t\sphinxhyphen{}elle la nature même de nos établissements ? La question peut paraître légitime (le bibliothécaire argue qu’il n’a pas été formé pour ça) est en droit de se demander si TOUT cela a bien sa place en bibliothèque.
On peut lui répondre que les bibliothèques sont depuis toujours le lieu des savoirs (littéraires, théorique et livresques), mais également celui des apprentissages.

Mais qui dit savoir\sphinxhyphen{}faire dit technique : le métier de bibliothécaire est\sphinxhyphen{}il celui d’un technicien ?
Les bibliothécaires ont toujours occupé des fonctions variées dans toutes sortes d’environnements et d’organisations. Ils ont joué le rôle d’enseignants, de facilitateurs, de collaborateurs, de chercheurs ou d’experts en technologie.
Mettre en place un makerspace est un prolongement naturel de la plupart de ces rôles traditionnels et la capacité à faire vivre un tiers lieu éducatif de ce type est un ajout précieux dans la trousse à outils de tous bibliothécaires orientés vers les services ou la formation.
\begin{itemize}
\item {} 
les bibliothèques sont passées d’une logique de conservation et d’accès à l’information (elles collectionnent à une logique de création et de partage facilité par les technologies actuelles ;

\item {} 
les bibliothèques ont toujours eu pour fonction de démocratiser l’accès à des ressources ou des technologies rares et coûteuses. C’est le cas aujourd’hui de technologies comme l’impression 3D ou la réalité virtuelle. On trouvait des machines à écrire dans de nombreuses bibliothèques publiques dans les années 1950.

\item {} 
les bibliothèques sont aujourd’hui des community hubs, des plateformes de rencontres et d’échanges entre groupes d’usagers que leurs centres d’intérêt communs rassemblent.

\end{itemize}

On redéfinit ce qu’est une bibliothèque aujourd’hui : des espaces orientés vers la sociabilité et la collaboration, des services qui font la place à la participation des usagers, des collections plus larges avec avec parfois l’accès à des outils ou encore le prêt d’instruments en complément des ressources documentaires traditionnelles.

\sphinxurl{http://www.abf.asso.fr/4/139/434/ABF/commission-fablab-presentation?p=2} (p. 113)

Un des projets, la borne de rétrogaming: \sphinxurl{https://cyrzbib.net/2017/02/04/operation-jaja-box-1ere-partie/}


\subsection{Une bibliothécaire formée au fablab (p. 117)}
\label{\detokenize{tutorials/work/livre/index:une-bibliothecaire-formee-au-fablab-p-117}}
Le Fablab propose deux formations diplômantes: initiation à la fabrication numérique et facilitateur permettant de devenir fabmanager\sphinxhyphen{}euse.

Au programme du D.U.
\begin{itemize}
\item {} 
des cours d’initiation à des logiciels et des machines ou outils : programmation Arduino, modélisation 2D et 3D, impression 3D, découpe vinyle, fraisage numérique, découpe et gravure laser ;

\item {} 
des sensibilisations : à l’accueil bienveillant, à la communication, à la gestion et l’entretien d’un parc machine, au droit de la propriété intellectuelle, à l’importance de la documentation, aux modèles économiques possibles… ;

\item {} 
des rencontres à co\sphinxhyphen{}organiser avec des acteur\sphinxhyphen{}rices oeuvrant dans tout type de lieu de fabrication numérique (fablab, hackerspace, makerspace…) et d’environnement (association, entreprise, établissement scolaire…) pour mieux comprendre le fonctionnement et la diversité de cet écosystème :

\item {} 
des ateliers à mettre en place pour transmettre nos apprentissages et se confronter à l’animation auprès d’un public (chose plutôt aisée pour moi, puisque l’animation est une activité classique en médiathèque) ;

\item {} 
des projets collectifs où l’ont apprend à faire ensemble et à s’entraider, chacun\sphinxhyphen{}e contribuant selon ses appétences et facilités ;

\item {} 
le développement d’un projet personnel tout au long du cursus ;

\item {} 
une implication pour visiter des lieux ;

\item {} 
un travail de documentation (formation, stage, projet personnel, visites).

\end{itemize}

L’esprit Lab:
\begin{itemize}
\item {} 
L’autonomie

\item {} 
La confiance

\item {} 
La souplesse

\item {} 
L’apprentissage par le faire

\item {} 
La pensée réseau (s’appuyer sur une communauté)

\item {} 
L’ouverture (accueil du public dans toute sa diversité, l’ouverture d’esprit, la recherche du dialogue, l’écoute active, l’attention portée à autrui)

\end{itemize}


\subsection{Les Mallapixels : un dispositif mobile de formation (p. 125)}
\label{\detokenize{tutorials/work/livre/index:les-mallapixels-un-dispositif-mobile-de-formation-p-125}}
Le Mallapixels est un laboratoire de fabrication itinérant et artistique à destination des acteurs culturels du Val\sphinxhyphen{}de\sphinxhyphen{}Marne, principalement les établissements de lecture publique, qui leur permet de s’initier puis de développer leur créativité numérique.
Chaque outil/matériel mis à disposition est toujours associé à une intention artistique.

Ce laboratoire artistique itinérant représente une collection de 26 objets insolites qui permettent de porter un regard actif sur la création artistique numérique. Ce matériel et ces objets numériques sont mis gratuitement à disposition des bibliothèques du Val\sphinxhyphen{}de\sphinxhyphen{}Marne par convention.
Ce prêt est accompagné de moments de formation, appelés les Fabriques. Ces formations ont pour objectif d’accompagner les médiathèques dans l’appropriation des outils numériques en construisant ensemble des scénarii d’apprentissage, c’est aussi la transmission des expériences de chacun autour de la médiation.

Le réseau des Mallapixels s’est constitué en 2015, et nous nous appuyons sur celui\sphinxhyphen{}ci pour organiser les Fabriques au sein des médiathèques ou dans les locaux de la Direction de la Culture à Créteil pouvant accueillir entre 10 à 12 personnes.
Les fabriques permettent de découvrir, de s’initier aux outils et de pouvoir repartir avec pour une durée maximale de trois mois. Cela laisse le temps de s’approprier l’outil et d’organiser une médiation sur plusieurs ateliers.

À ce jour, nous avons mis en place des Fabriques autour des thématiques suivantes: la table mashup, l’impression 3D, l’escape game, la boîte à histoire (Arduino), la découpeuse vinyle, la réalité augmentée (Aurasma), la réalité virtuelle (casque 3D), Dualo Touch.


\subsection{Le problème avec la documentation (p. 129)}
\label{\detokenize{tutorials/work/livre/index:le-probleme-avec-la-documentation-p-129}}
Les points communs entre les fablabs et bibliothèques: partage \sphinxhyphen{} savoir \sphinxhyphen{} documentation.

Il reste difficile d’insérer l’action de documenter au sein d’un projet. Les utilisateurs ont du mal à le faire. Prendre du recul et le temps de documenter peut paraître à contre\sphinxhyphen{}courant de l’activité créatrice.

C’est inscrit noir sur blanc dans la Charte des Fablab du MIT: « Contribuer à la documentation et aux connaissances des autres.

Le bibliothécaire connaît mieux que quiconque cet enjeu. Le professionnel des bibliothèques pourrait faire valoir son expertise dans le domaine de la documentation des projets développés.

Cette documentation une fois partagée est une source d’inspiration pour tous les makers à travers le monde. Des projets bien documentés sont une ressource essentielle pour la reproduction des projets et leur évolution future.

En bibliothèque, cela signifie aller chercher de l’information déjà existante alors qu’en fablab, il faut créer l’information.

La documentation n’est jamais finie. C’est un Work In Progress localement et via les réseaux. Les temps de la documentation sont à prendre en compte aussi. La construction de la documentation pour présenter le lieu, ses ateliers, ses créations, ses projets sont des temps différents (avant, pendant, après les temps de médiation).

Le wiki est un socle documentaire commun du lieu. En traitant et organisant les données du lieu, il lie et structure l’information, il rend tangible notre intelligence collective à l’oeuvre. Il nous aide à la création de nouveaux savoirs.

Le wiki du \sphinxhref{http://carrefour-numerique.cite-sciences.fr/fablab/wiki/doku.php?id=index}{Carrefour Numérique} ou celui du fablab de \sphinxhref{http://valby.copenhagenfablab.dk/projects}{Copenhague} sont des exemples inspirant.

Vers un modèle: curation (sources), co\sphinxhyphen{}création, participation et partage de la documentation créer à l’issue du projet.


\subsection{Le fabdocumentatliste est\sphinxhyphen{}il le nouveau bibliothécaire ? (p. 135)}
\label{\detokenize{tutorials/work/livre/index:le-fabdocumentatliste-est-il-le-nouveau-bibliothecaire-p-135}}
Au sein des fablabs, les usagers de ce type de lieu ont plutôt tendance à diffuser leurs savoirs et leurs expériences par l’oralité, au cours d’ateliers d’initiation ou d’échanges informels lors de rencontre sur place, que par l’écrit.

Pourtant nous pouvons y voir un paradoxe, sachant que les usagers sont, a priori plutôt à l’aise avec le numérique.
\begin{itemize}
\item {} 
\sphinxhref{https://wikifab.org/wiki/Accueil}{Projet Wikifab}

\end{itemize}

Bien documenter un projet est une activité très intéressante, mais très chronophage. On constate qu’il faut autant de temps, voire plus que pendant la phrase de fabrication elle\sphinxhyphen{}même.

Une documentation bien faite et visible apporte de nombreux avantages:
\begin{itemize}
\item {} 
permettre de mieux s’accaparer et d’approfondir son projet ;

\item {} 
faire connaître son projet au plus grand nombre et donc le valoriser ;

\item {} 
promouvoir de façon originale les activités du lieu ;

\item {} 
piquer la curiosité des gens, provoquer commentaires et rencontres.

\end{itemize}

Un poste tournant de documentaliste: fabdocumentaliste.
Ils établissent une fiche de poste, la mission globale est: promouvoir la documentation et la contribution auprès des usagers du fablab.
\begin{itemize}
\item {} 
Mission 1: s’informer sur les projets en cours entrepris dans le lieu et à l’extérieur ;

\item {} 
Mission 2: accompagner et orienter les usagers dans leurs démarches de documentation ;

\item {} 
Mission 3: organiser des événements et des actions autour de la documentation ;

\item {} 
Mission 4: organiser et diffuser la documentation auprès de la communauté locale et des communautés extérieures.

\end{itemize}


\subsection{Le design au service de la documentation des activités (p. 141)}
\label{\detokenize{tutorials/work/livre/index:le-design-au-service-de-la-documentation-des-activites-p-141}}\begin{itemize}
\item {} 
\sphinxhref{https://latelier-des-chercheurs.fr/outils/dodoc}{do•doc}

\end{itemize}

« Conçu pour documenter et créer des récits à partir d’activités pratiques, do•doc (prononcer doudoc) est un outil composite, libre et modulaire, qui permet de capturer des médias (photos, vidéos, sons et stop\sphinxhyphen{}motion), de les éditer, de les mettre en page et de les publier. Son aspect composite permet de le reconfigurer de manière à ce qu’il soit le plus adapté possible à la situation dans laquelle il est déployé. »


\bigskip\hrule\bigskip



\section{Développer une offre de service et d’atelier}
\label{\detokenize{tutorials/work/livre/index:developper-une-offre-de-service-et-d-atelier}}

\subsection{La recette (magique) pour inventer son Espace de Création Numérique en bibliothèque (p. 151)}
\label{\detokenize{tutorials/work/livre/index:la-recette-magique-pour-inventer-son-espace-de-creation-numerique-en-bibliotheque-p-151}}\begin{itemize}
\item {} 
L’usager est au centre : partir des besoins sur le terrain et des usages observés

\item {} 
Ouvrir plus, pas que les horaires ! Décloisonner les structures et les méthodes.

\item {} 
Coproduire : faire confiance à l’intelligence collective, lâcher prise

\item {} 
L’impact : innover c’est répondre à des problématiques

\item {} 
Expérimenter différents dispositifs dédiés innovants co\sphinxhyphen{}construits : des espaces mobiles, ponctuels ou pérennes

\item {} 
Créer une communauté d’intérêts : partager sur place et en ligne

\end{itemize}


\subsection{Combien ça coûte ? (p. 155)}
\label{\detokenize{tutorials/work/livre/index:combien-ca-coute-p-155}}
Pour un fablab éducatif, l’investissement est de 2000 dollars. Ce fablab spécifique se distingue par ses objectifs pédagogiques visant à amener les gens à apprendre la fabrication numérique.

Fablab Facotry propose un pack pour 5000 euros:
\begin{itemize}
\item {} 
Imprimante 3D: \sphinxhref{https://3dprinter.sindoh.com/en/product/dp201}{Sindoh DP201}

\item {} 
Découpe vinyle: \sphinxhref{https://www.lafourmicreative.fr/scanncut/125683-scanncut-sdx1200-brother--4977766792011.html?gclid=CjwKCAjw4MP5BRBtEiwASfwAL9YbSr\_OUa1mzkyINQSmWugApHjArUFRf6bOpsIuLQUU5vlqamSnmBoCS5oQAvD\_BwE}{Brother ScanNCut sdx 1200}

\item {} 
Machine à coudre: \sphinxhref{https://www.cdiscount.com/electromenager/repassage-couture/brother-machine-a-coudre-electronique-fs40-40/f-1101502-brotherfs40.html?idOffre=-1\&cid=search\_pla\&cm\_mmc=PLA!COR!PEM!CD!1040002306!m102400825\_pBROTHERFS40\_l9056437\_tpla-770416936231\_\&gclid=CjwKCAjw4MP5BRBtEiwASfwAL\_2JoiSIa5Ohub6ERoXiYcN1jmMfrEwSPyjtguOxe\_LXoAdxxMedFRoCDoQQAvD\_BwE}{Brother Innov\sphinxhyphen{}is 15}

\item {} 
Micro Ordinateur programmable: \sphinxhref{https://microbit.org/}{Micro:bit}

\item {} 
Kit Steam: \sphinxhref{https://samlabs.com/us/}{SAM Labs}

\item {} 
Kit code and robotique: \sphinxhref{https://strawbees.com/?wgu=280085\_206617\_1597087321366\_8d7f380841\&wgexpiry=1604863321\&source=webgains\&siteid=206617}{Strawbees}

\end{itemize}


\subsection{L’équipement des fablabs (p. 159)}
\label{\detokenize{tutorials/work/livre/index:l-equipement-des-fablabs-p-159}}
Description complète des appareils page 159.


\subsubsection{L’imprimante 3D}
\label{\detokenize{tutorials/work/livre/index:l-imprimante-3d}}
Une des stars des fablabs et des espaces de création numérique en bibliothèque. Les logiciels de modélisation pour débuter son Doodle3D, Tinkercad et Sketchup.
Il est possible de sauter cette étape en téléchargeant des fichiers 3D sur Thingiverse ou Cults par exemple.

Ensuite un logiciel va procéder au tranchage en générant un fichier Gcode, celui\sphinxhyphen{}ci est transféré à l’imprimante par câble USB ou carte SD.

La machine chauffe à environ 200°C afin de faire fondre le filament (ABS, PLA) qui sera déposé couche après couche pour fabriquer l’objet.

\sphinxstylestrong{Point faible:} leur fonctionnement est relativement lent, il faut environ 1 heure pour fabriquer un cube de 5 cm. C’est une machine parfois capricieuse et il n’existe pas vraiment de machine qui garantit zéro raté.

\sphinxstylestrong{Idées:} c’est un outil très polyvalent, il permet de développer des projets autour de la modélisation, du DIY (création de porte\sphinxhyphen{}clés, bijoux) de la robotique (impression pièce) ou encore de l’architecture (création de maquettes), etc.


\subsubsection{Le stylo 3D}
\label{\detokenize{tutorials/work/livre/index:le-stylo-3d}}
Les stylos 3D sont un partenaire pour les ateliers d’impression 3D, ils s’avèrent utiles pour « meubler » en proposant des animations complémentaires pendant l’impression.

\sphinxstylestrong{Point faible:} il ne faut pas s’imaginer que les stylos 3D permettent véritablement de dessiner dans les airs. Tout comme l’impression 3D, c’est un outil qui possède ses propres contraintes et nécessite une certaine prise en main.


\subsubsection{La découpe laser}
\label{\detokenize{tutorials/work/livre/index:la-decoupe-laser}}
Un adage des fablabs consiste à dire que les usagers se rendent au lab car ils ont entendu parler de l’impression 3D, mais utilisent majoritairement la découpe laser. En effet, ces machines sont beaucoup plus rapides et précises que les imprimantes 3D. En effet, ces machines sont beaucoup plus rapides et précises que les imprimantes 3D.
Les découpeuses laser sont essentiellement utilisées pour découper des plaques de bois et de plexiglas d’épaisseur variable.
Étant donné son rendement important, la découpeuse est souvent employée dans la réalisation de maquettes ou prototypes. Les pièces découpées sont alors collées ou assemblées ensemble avant d’aboutir au résultat final.

L’utilisation d’un découpeuse laser débute par un travail de modélisation en deux dimensions (plan en vectoriel). Une fois ce travail de modélisation terminé, le plan est envoyé à la machine qui découpe selon les formes souhaitées.
La découpeuse laser peut également être utilisée en mode gravure. Cette fonctionnalité permet d’abraser en surface afin de créer des motifs très fins.

\sphinxstylestrong{Inconvénients:} ce sont des outils coûteux (de 15 à 40 000 environ), elles sont donc réservées à des ECN de taille critique. De plus, elles nécessitent un entretien relativement rigoureux et parfois complexe. Les fumées issues de la combustion des matériaux impliquent l’utilisation d’un filtre ce qui pose un certain nombre de contraintes en termes d’espaces et d’installation. L’utilisation de ces machines doit donc être particulièrement encadrée.
Malgré ces inconvénients, la découpeuse laser reste la reine des fablabs.

\sphinxstylestrong{Ressources:} le wiki du Carrefour Numérique est particulièrement intéressant concernant l’utilisation de la découpe laser dans un ERP.


\subsubsection{Plotter de découpe ou découpeuse vinyle}
\label{\detokenize{tutorials/work/livre/index:plotter-de-decoupe-ou-decoupeuse-vinyle}}
Le principe est le suivant: on modélise une forme en deux dimensions et la machine découpe une plaque de matériau en suivant les tracés définis par l’utilisateur. La technique de découpe est mécanique, c’est\sphinxhyphen{}à\sphinxhyphen{}dire qu’une lame est utilisée pour fendre la matière choisie. Les matériaux sont multiples à commencer par le vinyle adhésif, le papier, le carton ou encore le floc, destiné à être transféré sur le textile. Les plotters de découpe permettent ainsi de réaliser des éléments de signalétique, des pop\sphinxhyphen{}up, des marque\sphinxhyphen{}pages ou encore la personnalisation de tote\sphinxhyphen{}bags.


\subsubsection{CNC}
\label{\detokenize{tutorials/work/livre/index:cnc}}
La CNC est dotée d’une fraise qui tourne à une vitesse élevée afin de venir couper ou entamer des matériaux comme le bois ou le métal afin d’obtenir le résultat escompté. Il peut travailler à partir d’un fichier 2D ou 3D en fonction du résultat souhaité.

C’est l’une des rares machines des fablabs qui permettent de travailler à une grande échelle.
Ce type de machines est très coûteux, plus de 25 000 euros.

\sphinxstylestrong{Inconvénient:} leur utilisation doit faire l’objecte d’une attention particulière, les mouvements de la fraise et les projections pouvant être dangereux. Un aménagement spécifique est nécessaire de façon à isoler la machine et limiter les désagréments sonores et la dispersion de la poussière dans l’ECN.

\sphinxstylestrong{Ressources:} il est possible de récupérer directement en ligne des plans de fabrication open source via des plateformes telles qu’OpenDesk.


\subsubsection{L’électronique}
\label{\detokenize{tutorials/work/livre/index:l-electronique}}
L’électronique dans les ECN repose fréquemment sur l’utilisation de cartes programmables telle qu’Arduino ou MakeyMakey. À ces cartes s’ajoutent de multiples composants que l’on peut regrouper en deux catégories : les capteurs et les actionneurs. Les capteurs permettent à la carte programmable de recevoir des informations de l’extérieur. Il peut s’agir de boutons, de capteurs de son ou de distance. Les actionneurs permettent à la carte programmable d’interagir avec le monde extérieur. Il peut s’agir de moteurs, de LED, de buzzer, etc.

Ces cartes et leurs composants peuvent donc être programmés pour réaliser une infinité de projets plus ou moins technique. Il peut s’agir de fabriquer des robots, un compteur pour la fréquentation de la bibliothèque, une bibliobox, etc.


\subsection{La Fabrique : un espace de fabrication numérique intégré au projet de l’établissement (p. 173)}
\label{\detokenize{tutorials/work/livre/index:la-fabrique-un-espace-de-fabrication-numerique-integre-au-projet-de-l-etablissement-p-173}}
La fabrique peut accueillir huit personnes maximum lors des animations. Elle est mise en avant via des ateliers récurrents animés par deux bibliothécaires, formés sur le tas.
Deux ateliers sont proposés, tous les quinze jours. Un le mardi soir, un le samedi après\sphinxhyphen{}midi. Ces ateliers de deux heures se focalisent sur une machine et sur la conception d’un objet. Ils sont à destination d’un public ado\sphinxhyphen{}adulte, et son gratuit.
Des animations ont lieu pendant les vacances scolaires, à destination d’un public jeunesse. De la même façon, chaque atelier se centre sur une machine, mais avec des usages plus ludiques que pour les adultes.

Les machines sont également accessibles en dehors des temps d’animations. Pour cela, les habitants doivent avoir un Permis Machine. L’obtention de ces permis se fait en ayant suivi un atelier avec les bibliothécaires, puis en répondant à un questionnaire en ligne: \sphinxurl{https://madeinlafabrique.wordpress.com/}
seuls les consommables sont à fournir par l’utilisateur. L’accès se fait pour le moment sur des créneaux identifiés qui correspondent à des moments de faible et moyenne affluence, où les bibliothécaires sont plus disponibles pour accompagner l’utilisateur.


\subsubsection{Retour après un an}
\label{\detokenize{tutorials/work/livre/index:retour-apres-un-an}}\begin{itemize}
\item {} 
L’imprimante 3D a créé des vocations et de nombreux utilisateurs ont acheté la leur et partage des astuces.

\item {} 
Le scanner 3D s’est révélé plus délicat à utiliser et n’est que très peu sollicité.

\item {} 
La découpeuse a rencontré un vif succès, surtout auprès d’un public de scrapbookers pour qui la pratique du papier découpé est déjà ancrée.

\item {} 
La brodeuse est la machine qui a eu le plus de succès, et qui est paradoxalement la plus complexe à utiliser.

\end{itemize}


\subsection{Le labo : le projet de laboratoire d’innovation numérique de la future médiathèque communautaire de Sainte\sphinxhyphen{}Geneviève\sphinxhyphen{}des\sphinxhyphen{}Bois (p. 177)}
\label{\detokenize{tutorials/work/livre/index:le-labo-le-projet-de-laboratoire-d-innovation-numerique-de-la-future-mediatheque-communautaire-de-sainte-genevieve-des-bois-p-177}}
Notre projet nu;2riaue se décline en quatre axes, qui ont d’emblée été liés avec les usages que nous retrouvons déjà au dessin des bibliothèques :
\begin{itemize}
\item {} 
Créer: participation aux animations utilisant le numérique: utiliser les consoles de l’esoace Jeu vidéo, apprendre à maîtriser outils et logiciels, créer/fabriquer en groupe des objets avec une imprimante 3D, des robots, mais également des oeuvres de l’esprit (atelier d’écriture, réalisation de vidéos, création de jeux vidéo, création d’images 3D et d’hologrammes, création sonore, graphiques, plastiques, etc).

\item {} 
Apprendre:des ateliers réguliers sur l’impression 3D, logiciel de montage vidéo, etc.

\item {} 
Échanger: la médiation autour des outils numériques se placera dans un contexte d’échanges et de troc des compétences. Les tournois de jeux vidéo seront l’occasion d’échanges entre passionnés et grands débutants, mais également avec d’autres équipes jouant dans d’autres bibliothèques.

\item {} 
Se former: la future médiathèque proposera des formations régulières sur l’utilisation des outils informatiques (création d’une adresse mail, première utilisation d’un ordinateur, d’une tablette) ou sur des compétences plus avancées (formation sur la protection de la vie privée sur internet, logiciels libres, etc).

\end{itemize}


\subsection{La Capsule : un espace pédagogique et créatif de la BU du Havre (p. 185)}
\label{\detokenize{tutorials/work/livre/index:la-capsule-un-espace-pedagogique-et-creatif-de-la-bu-du-havre-p-185}}
Trois zones dans le learning lab:
\begin{itemize}
\item {} 
une zone atelier: stations assises, debout et permettre des déplacements fréquents :

\item {} 
un espace pédagogique pour le travail en groupe, le partage d’information :  tables et chaises sur roulettes pour 25 à 30 personnes ;

\item {} 
un espace convivialité pour des postures détendues, du travail informel, des réunions d’équipe pédagogique, des pauses.

\end{itemize}

Donner un nom (la Capsule) a permis de communiquer sur l’espace et les activités associées, et de définir une identité visuelle avec le graphiste de l’université.

L’espace est prisé pour des enseignements variés, des rencontres associatives, des ateliers Wikipedia organisés entre pairs, des moments de vulgarisation scientifique, de la formation continue, des speeddating entrepreneuriaux, les réunions des bibliothécaires, des escapes games, des cafés pédagogiques en lien avec des MOOCs.


\subsection{Le bibliofab : faire entrer le monde des makers dans les bibliothèques (p. 193)}
\label{\detokenize{tutorials/work/livre/index:le-bibliofab-faire-entrer-le-monde-des-makers-dans-les-bibliotheques-p-193}}
Le BiblioFab est un dispositif mobile dont l’objectif est de rendre accessible l’univers des fablabs au plus grand nombre. Por se faire, le BiblioFap rempli trois fonctions : découvrir, fabriquer et exposer.
\begin{itemize}
\item {} 
Bibliothèque: des ouvrages sont à disposition pour que chacun puisse découvrir la culture maker et s’initier à la fabrication numérique ;

\item {} 
Atelier: comme dans un fablab, des ressources (machines, tutoriels) permettent de réaliser vos projets individuels et collectifs ;

\item {} 
Galerie: des espaces sont prévus pour exposer les créations réalisées et pour inspirer les autres utilisateurs.

\end{itemize}

Le BiblioFab répond à trois objectifs : présenter des projets réalisés dans les fablabs ; documenter les techniques de fabrication numérique ; offrir et ressources pour s’initier à la culture maker.
\begin{itemize}
\item {} 
Atelier: un dispositif permettant aux usagers de comprendre et de s’initier aux techniques de fabrication numérique et notamment de l’impression 3D ;

\item {} 
Expo: une sélection documentée d’objets produits dans les fablabs (prothèse de main, bijou connecté, objets d’art, objets upcyclés, etc) ;

\item {} 
Ressources: des ouvrages sélectionnés par les bibliothécaires, une tablette présentant une version numérique des tutoriels et des fiches explicatives de l’exposition téléchargeable en WIFI via une Bibliobox sur un nano ordinateur Raspberry Pi.

\end{itemize}

Une imprimante et un ordinateur sont à disposition des bibliothécaires afin d’animer des ateliers d’impression 3D. Des initiations sont donc proposées aux usagers qui, une fois formés, pourront développer leurs propres projets.


\subsection{L’atelier : un laboratoire de création au coeur de la médiathèque (p. 199)}
\label{\detokenize{tutorials/work/livre/index:l-atelier-un-laboratoire-de-creation-au-coeur-de-la-mediatheque-p-199}}
En 2015, une personne sur deux entrant à la médiathèque n’y venait plus pour emprunter. Travailler, se former, jouer, créer, découvrir était devenue des usages légitimes du lieu. Le bâtiment vieillissant, les espaces peu adaptés à leur utilisation réelle, l’évolution des usages, des outils et de l’offre nécessitaient l’écriture d’un projet de service actualisé.

L’Atelier a donc imaginé comme un espace qui permettrait de brasser les cultures, les outils et les techniques dans une double perspective:
\begin{itemize}
\item {} 
organiser l’éveil aux techniques plurielles de création à travers des outils et la transmission de savoir\sphinxhyphen{}faire. Ces techniques de création étant relative aux collections que nous pouvions par ailleurs proposer ;

\item {} 
permettre de porter des projets créatifs aboutis en ouvrant cet espace aux habitants et aux artistes pour mettre à leur disposition des outils de qualité professionnelle.

\end{itemize}

Création de trois stations de travail:
\begin{itemize}
\item {} 
une station dédiée au son et à la musique assistée par ordinateur ;

\item {} 
une station dédiée à la vidéo et à la photographie ;

\item {} 
une station dédiée au graphisme et à la modélisation.

\end{itemize}

La médiathèque a ainsi pu défendre l’intégration de cet atelier avec trop arguments forts:
\begin{itemize}
\item {} 
donner gratuitement accès à un espace doté d’outils coûteux permettant de favoriser et d’encourager les projets créatifs sur la ville (des outils ordinaires équipés de logiciels libres n’auraient pas permis de lisser les inégalités d’équipement au sein de la population) ;

\item {} 
en investissant dans un tel équipement, la ville se dotait d’un studio de création autonome pour mener à bien toutes sortes de projets créatifs à moindre coût ;

\item {} 
un financement important fut attribué sous forme de donations par la DRAC et la région à hauteur de 80\% pour la partie numérique.

\end{itemize}

Matériel:
\begin{itemize}
\item {} 
station image et vidéo : PC puissant, deux écrans art graphique, imprimante photo A3+, scanner photo, appareil reflex + zoom 24\sphinxhyphen{}70 mm, casque VR ;

\item {} 
station graphisme et modélisation : ordinateur hybride tablette graphique, plotter de découpe, imprimante 3D, presse à chaud ;

\item {} 
station son : PC, clavier midi, platine DJ, enceintes DJ, zoom son, micro ;

\item {} 
équipement mobile interactif : 14 tablettes tactiles, deux chariots de recharge, dix PC portables, un chariot de recharge, des cartes Arduino, Makey Makey, touch board, Raspberry pi.

\end{itemize}

Animations:
\begin{itemize}
\item {} 
sept animations hebdomadaires (écriture, photographie, graphisme, dessin vectoriel, vidéo, M.A.O, art du papier, découpent vinyle, flocage, impression 3D, réparation informatique, jeux vidéo, sensibilisation au développement durable, etc) ;

\item {} 
un stage de quatre jours toutes les vacances scolaires (exposition photo, construction d’une borne d’arcade, réalisation de courts métrages, etc) ;

\item {} 
prestations et interventions extérieures au fil de la saison (plasticiens, illustrateurs, associations de la ville, etc.).

\end{itemize}


\subsection{Ce que les bibs apportent aux labs (p. 207)}
\label{\detokenize{tutorials/work/livre/index:ce-que-les-bibs-apportent-aux-labs-p-207}}
Besoin des fablabs:
\begin{itemize}
\item {} 
espace (couvert par les médiathèques)

\item {} 
utilisateur (couvert par les médiathèques)

\item {} 
outils

\end{itemize}

Les bibliothèques constituent ainsi un maillage des territoires ruraux et urbains sans équivalent.

De nombreuses bibliothèques permettent la tenue de repair café à l’occasion desquels un ou plusieurs usagers organisent un atelier dédié à la réparation d’objets cassés.

L’impression 3D ou la robotique sont l’occasion pour les bibliothèques de remplir leur fonction d’accès aux savoirs et à la connaissance, mais également de tester l’intérêt du public, tout en commençant à identifier les usagers qui pourraient être moteurs dans l’éventualité où l’établissement souhaiterait développer son offre de service en créant son propre ECN.

Réalisation de robots contrôlables par smartphone et capables de s’affronter les uns les autres dans un combat d’éclatage de ballons de baudruche en se basant sur des kits Open Source JJrobots: \sphinxurl{https://www.jjrobots.com/}

\sphinxstylestrong{Quatre années de Fablab en bibliothèque:} initialement créée dans le but de faire découvrir au plus grand  nombre l’impression 3D, Wheeldo s’est très largement appuyé sur les bibliothèques pour remplir sa mission. En s’immergeant dans le milieu de la lecture publique, les membres de l’association ont rencontré un milieu particulièrement propice à la culture maker, résolument tourné vers les pratiques novatrices et surtout, un indispensable prisme de diffusion des savoirs et connaissances. En évoluant aux côtés des bibliothécaires pendant quatre années, les fabmanageurs ont vu évoluer la posture des professionnels et des usagers. Leurs rôles se sont transformés au fur et à mesure que les bibliothécaires et les usagers évoluaient de la découverte d’une technologie semblant inaccessible (tant techniquement que financièrement) vers une prise de conscience de ce que ce type d’outil pourrait apporter à leur quotidien. Les interventions de la micro\sphinxhyphen{}usine mobile ont alors commencé à s’éloigner des animations des origines en place d’ECN. Maintenant que les médiathèques et la société en générale semblent davantage prêtes à s’approprier les techniques de fabrication numérique, la mission s’achève? C’est donc vers une nouvelle aventure que se tournent le\textasciicircum{}s membres de l’association et ses fabmanagers avec la création d’une nouvelle entité, Ici\sphinxhyphen{}et\sphinxhyphen{}lab, dont la mission sera désormais de « diffuser la culture maker et accompagner la création de fablabs dans les territoires ».


\chapter{Veille technologique été 2020 \sphinxhyphen{} Danyl}
\label{\detokenize{tutorials/work/veille/index:veille-technologique-ete-2020-danyl}}\label{\detokenize{tutorials/work/veille/index::doc}}

\section{Logiciels}
\label{\detokenize{tutorials/work/veille/index:logiciels}}\begin{itemize}
\item {} 
ZBrushCoreMini logiciel gratuit avec tous les outils essentiels pour la sculpture, il sera utilisé en sculpture organique.

\end{itemize}

\begin{sphinxadmonition}{note}{Note:}
Gratuit \sphinxurl{https://youtu.be/gxyMejEA0PE}
\end{sphinxadmonition}

\noindent\sphinxincludegraphics{{zbrush}.jpg}
\begin{itemize}
\item {} 
Godot 4.0 supportera cette année avec des avancés notable sur le rendu graphique. Il utilise Python. Le moteur pourrait être utilisé pour créer des jeux 3D: \sphinxurl{https://youtu.be/MbCVixMSDvo}

\end{itemize}

\begin{sphinxadmonition}{note}{Note:}
Gratuit: \sphinxurl{https://godotengine.org/}
\end{sphinxadmonition}
\begin{itemize}
\item {} 
Core, basé sur Unreal Engine 4, permet de créer des jeux rapidement, souvent orienté de tir. Les graphismes ressemblent à Fortnite. Nécessite une connexion internet pour être utilisé et de le mettre souvent à jour. Le logiciel propose beaucoup de ressources et de scripts (codes) déjà réalisés. \sphinxurl{https://youtu.be/2aTfknWRXAc}

\end{itemize}

Tutoriel sur la création d’un jeu multijoueur avec Core: \sphinxurl{https://youtu.be/-rIbd612sUM}

\begin{sphinxadmonition}{note}{Note:}
Gratuit: \sphinxurl{https://www.coregames.com/}
\end{sphinxadmonition}

\noindent\sphinxincludegraphics{{core}.jpg}


\section{Impression 3D}
\label{\detokenize{tutorials/work/veille/index:impression-3d}}\begin{itemize}
\item {} 
Elegoo Saturn, première grande imprimante résine abordable avec une vitesse de 1.5\sphinxhyphen{}2s par layer à 50 microns (plusieurs minutes pour la 3D Up Box+ à 200 microns): \sphinxurl{https://youtu.be/VNxcpQEBlWY}

\end{itemize}

\begin{sphinxadmonition}{note}{Note:}
Prix 400 dollars (sortie en octobre): \sphinxurl{https://www.elegoosaturn.com/products/elegoo-saturn-8-9-inch-4k-uv-lcd-3d-printer}
\end{sphinxadmonition}

\noindent\sphinxincludegraphics{{elegoo}.png}


\section{VR}
\label{\detokenize{tutorials/work/veille/index:vr}}\begin{itemize}
\item {} 
Pimax 8k+ une résolution 8 fois supérieures à l’Occulus Rift qui souffre d’une image très pixelisée et floue: 2*3840×2160 pour le Pimax 8k+ contre 2*1080×1200 pour l’Occulus Rift et un angle de vue bien supérieur à 200 degrés: \sphinxurl{https://youtu.be/gozLg1DHUiQ}

\end{itemize}

\begin{sphinxadmonition}{note}{Note:}
Prix 899 dollars: \sphinxurl{https://www.pimax.com/products/vision-8k-plus-withoutmas}
\end{sphinxadmonition}

\noindent\sphinxincludegraphics{{Pimax}.png}


\section{Jeux vidéo}
\label{\detokenize{tutorials/work/veille/index:jeux-video}}\begin{itemize}
\item {} 
Microsoft Flight Simulator: première simulation à intégrer un scan laser de la planète entière à quelques mètres près (inclus la MDE et votre maison): \sphinxurl{https://youtu.be/TYqJALPVn0Y}

\end{itemize}

\noindent\sphinxincludegraphics{{flight}.png}
\begin{itemize}
\item {} 
Car Mechanic Simulator propose aussi Motorcycle Mechanic Simulator pour la réparation de moto  et Biker Garage Mechanic Simulator pour la réparation de motos américaines.

\end{itemize}


\section{Simulation}
\label{\detokenize{tutorials/work/veille/index:simulation}}\begin{itemize}
\item {} 
F\sphinxhyphen{}GT Lite: premier cockpit pliable avec fauteuil, support levier de vitesse et position configurable: \sphinxurl{https://youtu.be/vMiLRCpc3uQ}

\end{itemize}

\begin{sphinxadmonition}{note}{Note:}
Prix 299 euros: \sphinxurl{https://www.nextlevelracing.com/products/f-gt-lite/}
\end{sphinxadmonition}

\noindent\sphinxincludegraphics{{f-gt_lite}.png}
\begin{itemize}
\item {} 
Alternative à la VR pour le simulateur, Samsung C49HG90DMU FreeSync (dalle incurvée): \sphinxurl{https://youtu.be/BPaL\_Mp9HK4}

\end{itemize}

\begin{sphinxadmonition}{note}{Note:}
Prix 899 euros: \sphinxurl{https://www.topachat.com/pages/detail2\_cat\_est\_peripheriques\_puis\_rubrique\_est\_w\_moni\_puis\_ref\_est\_in10111427.html}
\end{sphinxadmonition}

\noindent\sphinxincludegraphics{{wide}.jpg}


\subsection{Si évolution du projet du simulateur}
\label{\detokenize{tutorials/work/veille/index:si-evolution-du-projet-du-simulateur}}\begin{itemize}
\item {} 
AccuForce Pro V2: Le meilleur rapport qualité prix pour un volant à technologie Direct Drive, seul problème il est bruyant (attendre pour une version au dessus): \sphinxurl{https://youtu.be/g-XCq4fMekc}

\end{itemize}

\begin{sphinxadmonition}{note}{Note:}
Prix 999 dollars \sphinxurl{https://simxperience.com/products/accessories/accuforcesteering/accuforcepro.aspx}
\end{sphinxadmonition}

\noindent\sphinxincludegraphics{{accuforce}.jpg}
\begin{itemize}
\item {} 
Sim Lab GT\sphinxhyphen{}1 EVO: Le meilleur cockpit fixe et abordable actuel (pour un simulateur qui resterait souvent sur place, vissé dans un véhicule ou prêté longtemps): \sphinxurl{https://youtu.be/jpJdxDsYIrs}

\end{itemize}

\begin{sphinxadmonition}{note}{Note:}
Prix 399 euros \sphinxurl{https://sim-lab.eu/shop/product/gt-1-evo-sim-racing-cockpit-446?category=3\#attr=354}
\end{sphinxadmonition}

\noindent\sphinxincludegraphics{{gt1}.jpg}
\begin{itemize}
\item {} 
Les meilleures pédales haut de gamme abordables actuelles: \sphinxurl{https://youtu.be/SMwcFrTsKd4}

\end{itemize}

\begin{sphinxadmonition}{note}{Note:}
Prix 699 euros \sphinxurl{https://heusinkveld.com/products/sim-pedals/sim-pedals-sprint/?q=\%2Fproducts\%2Fsim-pedals\%2Fsim-pedals-sprint\%2F\&v=11aedd0e4327}
\end{sphinxadmonition}

\noindent\sphinxincludegraphics{{sprint}.png}

\sphinxstylestrong{Aperçu du simulateur complet avec l’écran ultra wide (camion aménagé):} \sphinxurl{https://youtu.be/7R4LmSNsMws}


\chapter{Imprimante 3D: 3D UP BOX+}
\label{\detokenize{tutorials/fabrication/3dupbox/index:imprimante-3d-3d-up-box}}\label{\detokenize{tutorials/fabrication/3dupbox/index::doc}}
Todo:
\begin{itemize}
\item {} 
Mettre photo passage filament

\item {} 
Preview dans printing pour voir les supports et temps. Les suports disparaissent quand on oriente le modèle (bug).

\item {} 
raft = radier

\item {} 
chercher où trouver temps d’impression

\item {} 
chercher si option poser au sol comme dans Cura

\item {} 
Montrer filament custom et comment sauvegarder

\end{itemize}

\noindent\sphinxincludegraphics{{customize}.png}

\noindent\sphinxincludegraphics{{bouton_edit}.png}

\noindent\sphinxincludegraphics{{parameters}.png}

\noindent\sphinxincludegraphics{{error_material}.png}

\noindent\sphinxincludegraphics{{remaining}.png}


\bigskip\hrule\bigskip



\section{Fonction}
\label{\detokenize{tutorials/fabrication/3dupbox/index:fonction}}
L’imprimante 3D vous permet de concevoir des pièces en plastique automatiquement à partir d’un fichier informatique.


\section{Liens}
\label{\detokenize{tutorials/fabrication/3dupbox/index:liens}}\begin{itemize}
\item {} 
\sphinxhref{https://www.tiertime.com/up-box-plus/}{Site officiel}

\item {} 
\sphinxhref{https://youtu.be/QgTA9QPbbdM}{Vidéo démarrage rapide}

\item {} 
\sphinxhref{https://3dprintingsystems.com/download/UP\_BOX+\_Manual\_4.8\_EN.pdf}{Documentation officielle}

\item {} 
{\color{red}\bfseries{}\textasciigrave{}Forum officiel \textless{}https://forum.tiertime.com/c/up\sphinxhyphen{}box\sphinxhyphen{}up\sphinxhyphen{}box/15\textasciigrave{}\_}

\item {} 
\sphinxhref{https://www.a4.fr/wiki/index.php?title=FAQ\_Imprimantes\_3D\_TIERTIME}{FAQ}

\end{itemize}


\section{Matériel}
\label{\detokenize{tutorials/fabrication/3dupbox/index:materiel}}
\noindent\sphinxincludegraphics{{upbox}.png}


\section{Logiciel}
\label{\detokenize{tutorials/fabrication/3dupbox/index:logiciel}}\begin{itemize}
\item {} 
\sphinxhref{https://s3-us-west-1.amazonaws.com/up3d/downloads/UP\_Studio\_x64\_2.6.49.627.zip}{Télécharger Up Studio} (logiciel pour préparer le modèle à l’impression)

\end{itemize}


\section{Récupérer un modèle sur internet}
\label{\detokenize{tutorials/fabrication/3dupbox/index:recuperer-un-modele-sur-internet}}\begin{itemize}
\item {} 
\sphinxhref{https://www.thingiverse.com/}{Thingiverse}

\end{itemize}

\begin{sphinxadmonition}{note}{Note:}
Les fichiers sont généralement compressés dans un fichier au format .zip, pour utiliser son contenu double\sphinxhyphen{}cliquez dessus puis glissez les fichiers à l’intérieur dans un dossier préalablement créé dans l’ordinateur.
Vous pouvez ne décompresser que les fichiers .stl, ce sont les modèles 3D.
\end{sphinxadmonition}


\section{Utilisation}
\label{\detokenize{tutorials/fabrication/3dupbox/index:utilisation}}

\subsection{Préparation du matériel}
\label{\detokenize{tutorials/fabrication/3dupbox/index:preparation-du-materiel}}
\begin{sphinxadmonition}{attention}{Attention:}
Todo parler des deux types de plateaux: UP Flex Board et Perf Board.
\end{sphinxadmonition}
\begin{itemize}
\item {} 
Installez le plateau en le plaquant contre celui en aluminium. Alignez les vis, puis poussez\sphinxhyphen{}le vers le fond avec vos mains à plat.

\end{itemize}

\noindent\sphinxincludegraphics{{vis}.png}
\begin{itemize}
\item {} 
Enlevez le couvercle magnétique sur le côté et mettez le filament. Si la bobine est épaisse, utilisez la prolongation de fixation avant de mettre la bobine. Ouvrez l’imprimante 3D par le dessus, enlevez le guide\sphinxhyphen{}fil de l’extrudeur, faites passer le filament à l’intérieur jusqu’à le voir ressortir puis mettez\sphinxhyphen{}le dans l’extrudeur à nouveau. Remettez le couvercle magnétique.

\end{itemize}

\begin{sphinxadmonition}{note}{Note:}
Utilisez de préférence du PLA car il est moins polluant que l’ABS.
\end{sphinxadmonition}

\sphinxstyleemphasis{Filament en place avec la prolongation de fixation:}

\noindent\sphinxincludegraphics{{filament}.png}

\sphinxstyleemphasis{Le filament ressort par le guide\sphinxhyphen{}fil et va dans l’extrudeur:}

\noindent\sphinxincludegraphics{{guide-fil}.png}

\sphinxstyleemphasis{Couvercle magnétique en place:}

\noindent\sphinxincludegraphics{{couvercle}.png}
\begin{itemize}
\item {} 
Connectez l’imprimante en USB, branchez l’alimentation. Appuyez sur le bouton à l’arrière pour la mettre sur On (symbole du trait).

\end{itemize}

\noindent\sphinxincludegraphics{{branchements}.png}
\begin{itemize}
\item {} 
Restez appuyé sur le bouton d’allumage plusieurs secondes (cela permet d’initialiser l’imprimante à sa position par défaut lors de l’allumage).

\end{itemize}

\noindent\sphinxincludegraphics{{power}.png}


\subsection{Réglages logiciel}
\label{\detokenize{tutorials/fabrication/3dupbox/index:reglages-logiciel}}\begin{itemize}
\item {} 
Lancez le logiciel \sphinxhref{file:///C:/Users/MEDIATHEQUE1/Documents/GitHub/test-readthedocs/docs/\_build/html/tutorials/fabrication/3dupbox/index.html}{Up Studio}, c’est un logiciel dans lequel vous importez vos modèles 3D afin de générer un fichier avec les instructions sur la fabrication de celui\sphinxhyphen{}ci par l’imprimante 3D.
Ce type de logiciel s’appelle un « Slicer » (découpeur) car il indique les couches et les trajectoires que devra effectuer l’imprimante pour déposer le plastique.

\end{itemize}

\begin{sphinxadmonition}{note}{Note:}\begin{itemize}
\item {} 
Au premier lancement du logiciel, il vous sera demandé de vous connecter à un compte, nous vous invitons à créer le vôtre. Cliquez à gauche sur \sphinxcode{\sphinxupquote{Account}} (compte), puis au centre sur \sphinxcode{\sphinxupquote{Sign Up}} (s’inscrire).

\item {} 
Lors de votre connexion, cochez \sphinxcode{\sphinxupquote{Auto Login}} pour vous reconnecter automatiquement au lancement du logiciel.

\end{itemize}
\end{sphinxadmonition}
\begin{itemize}
\item {} 
Calibrez le plateau (à faire impérativement après un transport de l’imprimante), pour cela cliquez sur l’icône \sphinxcode{\sphinxupquote{Calibration}} représentée par deux règles perpendiculaire, dans la fenêtre qui s’ouvre cliquez sur « Auto Level ».

\end{itemize}

\begin{sphinxadmonition}{attention}{Attention:}
TODO: pendant le calibrage, l’imprimante va mesurer la taille de la buse (extrudeur), si l’imprimante fait un bruit de percussion {[}..{]}
\end{sphinxadmonition}

\noindent\sphinxincludegraphics{{calibration}.png}
\begin{itemize}
\item {} 
Allez dans Maintenance et vérifiez que le Material type soit bien sûr PLA et que le Print Board soit celui qui est dans l’imprimante (Perf Board ou Up Flex Board).

\end{itemize}

\noindent\sphinxincludegraphics{{pla}.png}

\begin{sphinxadmonition}{note}{Note:}
Pour déboucher la buse, cliquez sur l’icône Maintenance puis sur Extrude, le filament va sortir de la buse, cliquez sur Stop et coupez le à la base avec des ciseaux. Pour remplacer le filament, cliquez sur Withdraw.
\end{sphinxadmonition}


\subsection{Préparer une impression sur UP Studio}
\label{\detokenize{tutorials/fabrication/3dupbox/index:preparer-une-impression-sur-up-studio}}\begin{itemize}
\item {} 
Chargez le modèle avec le bouton +. Vous pouvez ajouter un polygone de base qui vous est proposé ou importer un modèle 3D au format .stl ou .obj en cliquant sur Add 3D Model.

\end{itemize}

\noindent\sphinxincludegraphics{{3dmodel}.png}

\begin{sphinxadmonition}{note}{Note:}
Vous pouvez glisser/déposer le fichier dans la vue 3D aussi.
\end{sphinxadmonition}

Pour vous déplacer dans la vue 3D:
\begin{itemize}
\item {} 
le \sphinxstylestrong{clic gauche} permet de tourner dans la vue 3D

\item {} 
la \sphinxstylestrong{molette} sert à avancer ou reculer

\item {} 
le \sphinxstylestrong{clic droit} déplace la vue sur les côtés

\end{itemize}

La « roue » en haut à droite sert à déplacer (le moins utilisé), mise à l’échelle (le plus utilisé),

\noindent\sphinxincludegraphics{{roue}.png}

Le Bouton \sphinxcode{\sphinxupquote{Auto Place}} représenté par l’icône d’une croix placec le modèle à la hauteur du plateau.

\begin{sphinxadmonition}{note}{Note:}
Le modèle touche par défaut le plateau à son importation, mais il peut arriver qu’il ne soit plus en contact après une rotation.
\end{sphinxadmonition}

\noindent\sphinxincludegraphics{{auto_place}.png}

Pour doubler l’échelle, cliquez sur l’icône Echelle puis sur le numéro 2, pour diviser par 2 cliquez sur 0.5. Sur l’image en dessous l’échelle est doublée:

\noindent\sphinxincludegraphics{{doublescale}.png}

\begin{sphinxadmonition}{attention}{Attention:}
Lorsque vous faites une mise à l’échelle les dimensions sont changées sur 3 axes, en doublant l’échelle vous multipliez le temps d’impression par 2 sur l’axe X, par 2 sur l’axe Y et par 2 sur l’axe Z, vous augmenterez le temps de 2x2x2 donc l’impression prendra 8 fois plus de temps. Une impression qui prenait 1 heure en prendra 8.
\end{sphinxadmonition}

Et inversement, en cliquant sur 0.5 vous divisez par 8 le temps d’impression. L’échelle est donc le paramètre qui aura le plus d’incidence sur la durée de l’impression.

L’orientation va permettre d’éviter l’utilisation de supports, donc d’imprimer plus rapidement et de passer moins de temps à poncer le modèle pour lisser les surfaces.

Sur le modèle de gauche l’orientation nécessite beaucoup de supports, la tête est placée à la verticale et l’arrière du crâne et les deux oreilles ont besoin de maintien.
Celui de droite est orienté pour que l’arrière du crâne touche le plateau ainsi qu’une des deux oreilles.

\noindent\sphinxincludegraphics{{orientation}.png}

Un autre exemple, sur l’image de gauche le modèle n’a besoin d’aucun support et les surfaces seront plus lisses. Sur l’image de droite, beaucoup de supports seront nécessaire et la surface sera en « escaliers ».

\noindent\sphinxincludegraphics{{orientation2}.png}


\subsection{Lancer une impression}
\label{\detokenize{tutorials/fabrication/3dupbox/index:lancer-une-impression}}
Cliquez sur l’icône Print, des options s’afficheront avant l’impression:

\noindent\sphinxincludegraphics{{printsettings}.png}
\begin{itemize}
\item {} 
Layer Thickness: c’est l’épaisseur des couches d’impression, 100 microns servent aux impressions détaillées comme les figurines et 200 pour les pièces mécaniques. Plus l’impression est épaisse moins elle prendra de temps à imprimer.

\item {} 
Infill: c’est le remplissage, les pièces remplies sont celles qui sont soumises à des forces comme des engrenages.

\item {} 
Quality: c’est la vitesse de déplacement de la buse, cela affecte la précision du modèle.

\item {} 
Nozzle offset:

\item {} 
Unsolid Model: ferme un modèle s’il comporte des trous.

\item {} 
No Raft: si l’option est cochée cela désactive le support sous l’impression 3D pour renforcer l’adhérence. Une pièce fine et cylindrique par exemple aura besoin d’un raft. Raft se traduire par radeau.

\item {} 
No Support: désactive les supports. Concevoir et/ou orienter un modèle qui ne nécessite pas de support permet de gagner du temps pendant l’impression et après (découpe des supports, ponçage).

\end{itemize}

\begin{sphinxadmonition}{attention}{Attention:}
TODO: chercher Nozzle offset et Unsolid Model.
\end{sphinxadmonition}

\begin{sphinxadmonition}{attention}{Attention:}
une fois que cela fonctionnera, parler du mode preview, chercher si indicateur de durée.
\end{sphinxadmonition}

Cliquez sur Print (imprimer) pour démarrer l’impression.


\chapter{Imprimante 3D: Dagoma Neva Magis}
\label{\detokenize{tutorials/fabrication/magis/index:imprimante-3d-dagoma-neva-magis}}\label{\detokenize{tutorials/fabrication/magis/index::doc}}\begin{itemize}
\item {} 
\sphinxhref{https://www.dagoma3d.com/imprimante-3d-magis-dagoma}{Site officiel}

\item {} 
\sphinxhref{https://www.lesimprimantes3d.fr/forum/41-dagoma/}{Forum}

\end{itemize}


\section{Matériel}
\label{\detokenize{tutorials/fabrication/magis/index:materiel}}
\noindent\sphinxincludegraphics{{magis}.jpg}


\section{Logiciel}
\label{\detokenize{tutorials/fabrication/magis/index:logiciel}}\begin{itemize}
\item {} 
\sphinxhref{https://dist.dagoma3d.com/get/zip/CuraByDagoma/1568220765/6d41d9077db0874f55a6d89c0914a9f4}{Télécharger Cura by Dagoma} (logiciel pour préparer le modèle à l’impression)

\end{itemize}


\chapter{Photogrammétrie: Sense 3D V2}
\label{\detokenize{tutorials/fabrication/sense/index:photogrammetrie-sense-3d-v2}}\label{\detokenize{tutorials/fabrication/sense/index::doc}}

\section{Fonction}
\label{\detokenize{tutorials/fabrication/sense/index:fonction}}
Un scanner 3D capture la forme d’objets réel pour en faire des modèles 3D virtuels sur l’ordinateur.
A partir de ces fichiers il est ensuite possibe de concevoir de nouveles pièces afin de réparer et améliorer un objet.
Les objets scannés sont souvent utilisé dans les jeux vidéo moderne.


\section{Liens}
\label{\detokenize{tutorials/fabrication/sense/index:liens}}\begin{itemize}
\item {} 
\sphinxhref{https://fr.3dsystems.com/3d-scanners/sense-scanner}{Site officiel}

\item {} 
\sphinxhref{https://s3.amazonaws.com/dl.3dsystems.com/binaries/support/sense-scanner/Sense2\_UserGuide\_031519.pdf}{Guide utilisateur}

\item {} 
\sphinxcode{\sphinxupquote{Sense Track Assist}} (aide à la détection)

\end{itemize}


\section{Matériel}
\label{\detokenize{tutorials/fabrication/sense/index:materiel}}
\noindent\sphinxincludegraphics{{sense}.png}


\section{Logiciels}
\label{\detokenize{tutorials/fabrication/sense/index:logiciels}}\begin{itemize}
\item {} 
\sphinxhref{https://telecharger.freedownloadmanager.org/Windows-PC/3D-Systems-Sense/GRATUIT-2.2.0.240.html?ac1acbc}{Télécharger le logiciel 3D Systems Sense}

\end{itemize}


\section{Utilisation}
\label{\detokenize{tutorials/fabrication/sense/index:utilisation}}
Après avoir installé le logiciel 3D Systems Sense connectez le scanner sur un port USB.

\begin{sphinxadmonition}{note}{Note:}
Connectez le scanner sur un port USB 3.0 pour améliorer les performances, ces ports sont souvent en bleu.
\end{sphinxadmonition}

Vous devriez voir l’affichage de la caméra dans le logiciel. Choisissez le type d’objet à scanner entre Objet, Tête et Corps puis cliquez sur « Numériser » pour commencer.

\noindent\sphinxincludegraphics{{scan_type}.png}

\begin{sphinxadmonition}{note}{Note:}
Conseils pour scanner
\begin{itemize}
\item {} 
Tenez\sphinxhyphen{}vous à une distance d’environ 45 cm à 2 mètres de l’objet, selon la taille de l’objet qui est scanné.
\begin{itemize}
\item {} 
Si l’objet scanné devient blanc, c’est qu’il est mal détecté, cela peut\sphinxhyphen{}être parce qu’il est trop proche, parce qu’il est brillant, parce qu’il n’y a pas assez de points de repère (une sphère par exemple. Dans le dernier cas, vous pouvez ajouter une feuille de repère sous l’objet pour améliorer la détection: \sphinxcode{\sphinxupquote{Sense Track Assist}}

\end{itemize}

\end{itemize}
\end{sphinxadmonition}


\bigskip\hrule\bigskip


Vidéo explicative: \sphinxurl{https://youtu.be/1eWUaxq-oGg}

Solidify rempli les trous automatiquement.

\begin{sphinxadmonition}{attention}{Attention:}
Le logiciel 3D Systems Sense ne permet pas de retour arrière. Exporter votre modèle régulièrement avant une manipulation. Vous pourrez effectuer les mêmes tâches (en plus complexe) sur le logiciel de 3D Blender.
\end{sphinxadmonition}

\begin{sphinxadmonition}{note}{Note:}
Conseils pour scanner:
\begin{itemize}
\item {} 
Tenez\sphinxhyphen{}vous à environ 45 cm à 2 mètres de l’objet. La distance dépend aussi de la taille de ce qui est scanné.
\begin{itemize}
\item {} 
Si l’objet scanné devient blanc, c’est que vous êtes trop proche.

\end{itemize}

\item {} 
Vous pouvez faire plusieurs tours avec différents angles pour ajouter des données au scan et remplir les trous.

\item {} 
Scannez avec un angle à 45 degrés vers le bas pour le premier tour.

\item {} 
Pour une personne, pointez la caméra vers le visage.

\end{itemize}
\end{sphinxadmonition}

\begin{sphinxadmonition}{note}{Note:}
Chaque 30 degrés autour du modèle le logiciel va demander de rester fixe quelques secondes pour optimiser la capture.
\end{sphinxadmonition}

\begin{sphinxadmonition}{note}{Note:}
Éclairage en intérieur et dans plusieurs directions. Il faut le moins d’ombre possible.
\end{sphinxadmonition}

\begin{sphinxadmonition}{note}{Note:}
Le retour arrière n’est possible qu’avec les flèches, vous ne pouvez le faire avec Ctrl + Z.
\end{sphinxadmonition}

\noindent\sphinxincludegraphics{{retour}.png}

\begin{sphinxadmonition}{note}{Note:}
L’outil couper supprimer la zone la plus petite.
\end{sphinxadmonition}

Il faut cliquer sur Terminer pour exporter

Cliquer sur Numériser à gauche pour démarrer, positionnez bien la caméra pendant le compte à rebours pour ne pas capturer un autre objet.

\begin{sphinxadmonition}{attention}{Attention:}
Avec le niveau de détail maximal le scan peut rapidement perdre l’object.
\end{sphinxadmonition}

Réparation \textgreater{} Solidifier

\noindent\sphinxincludegraphics{{Solidification}.png}

Test:

\noindent\sphinxincludegraphics{{test}.png}


\section{Corriger son scan 3D sous Blender}
\label{\detokenize{tutorials/fabrication/sense/index:corriger-son-scan-3d-sous-blender}}
Importer le .obj

Faire un scale à 0.05. Ajoutez un cube et faites un scale et positionnez l’objet pour qu’il rentre à peu près dans le cube. Orientez le modèle pour qu’il soit à l’endroit. Puis faire un Apply en All Transforms.

Remesh à 0.05

Sculpt mode sans symmétrie

Smooth puis Inflate puis nouveau remesh

Continuer avec le inflate, smooth, snake hook et remesh pour reconstituer le modèle original


\chapter{Photogramétrie: Smartphone}
\label{\detokenize{tutorials/fabrication/smartphone/index:photogrametrie-smartphone}}\label{\detokenize{tutorials/fabrication/smartphone/index::doc}}
\sphinxurl{https://youtu.be/1D0EhSi-vvc}


\section{Matériel}
\label{\detokenize{tutorials/fabrication/smartphone/index:materiel}}

\section{Logiciels}
\label{\detokenize{tutorials/fabrication/smartphone/index:logiciels}}

\section{Utilisation}
\label{\detokenize{tutorials/fabrication/smartphone/index:utilisation}}

\chapter{Stylo 3D: 3Doodler Start}
\label{\detokenize{tutorials/fabrication/3doodler/index:stylo-3d-3doodler-start}}\label{\detokenize{tutorials/fabrication/3doodler/index::doc}}\begin{itemize}
\item {} 
\sphinxhref{https://learn.the3doodler.com/getting-started/start/}{Site officiel}

\item {} 
\sphinxhref{https://learn.the3doodler.com/resources/}{Idées de projets}

\end{itemize}


\section{Matériel}
\label{\detokenize{tutorials/fabrication/3doodler/index:materiel}}
\noindent\sphinxincludegraphics{{3doodler}.jpg}


\section{Logiciels}
\label{\detokenize{tutorials/fabrication/3doodler/index:logiciels}}

\section{Utilisation}
\label{\detokenize{tutorials/fabrication/3doodler/index:utilisation}}
Rechargez le 3doodler en le branchant par USB à un ordinateur ou la prise d’un chargeur de smartphone.
La diode va clignoter en orange pendant la recharge, une fois chargé à 100\% la diode restera allumé en orange.

\begin{sphinxadmonition}{note}{Note:}
Il faut compter environ 2h à 2h 30 pour recharger le 3doodler Start.
\end{sphinxadmonition}

Débranchez le 3doodler et mettez le sur On. La diode sera allumée en rouge pendant la chauffe, une fois à température elle sera allumé en verte.
Le stylo est prêt:

\noindent\sphinxincludegraphics{{pret}.png}

Mettez un filament spécialement pour le 3doodler.

\begin{sphinxadmonition}{attention}{Attention:}
Le stylot ne fonctionne pas avec le filament d’imprimantes 3D car la température du 3Doodler n’est pas assez élevé pour le faire fondre.
\end{sphinxadmonition}

\begin{sphinxadmonition}{attention}{Attention:}
Le plastique se consomme très rapidement et est onéreux. Il sera important de ne donner l’appareil qu’après avoir expliqué son fonctionnement et de surveiller son utilisation.
\end{sphinxadmonition}

Appuyez une fois sur le bouton orange pour pousser le plastique, une autre fois pour arrêter.

\noindent\sphinxincludegraphics{{extrusion}.png}

Appuyez deux fois pour retirer le plastique.

\begin{sphinxadmonition}{note}{Note:}
Le plastique est difficile à retirer, il vous faudra secouer le stylo pour le faire tomber.
\end{sphinxadmonition}

Pour créer des formes en volume il faut compter environ 35 secondes pour que le plastique devienne solide.
Vous pouvez créer des surface à plat puis les élevé et attaché une autre surface.

Si le plastique ne sort pas, vous pouvez le pousser au début avec une tige métallique.

\noindent\sphinxincludegraphics{{pousser_filament}.png}


\chapter{Découpeuse vinyle: Silhouette Curio}
\label{\detokenize{tutorials/fabrication/curio/index:decoupeuse-vinyle-silhouette-curio}}\label{\detokenize{tutorials/fabrication/curio/index::doc}}
\begin{sphinxadmonition}{note}{Note:}
Le papier doit dépasser les bordures en haut à gauche. Renforcer l’adhérence en pressant avec la main.
\end{sphinxadmonition}

Todo: PixScan: \sphinxurl{http://www.silhouettefr.fr/silhouette\_pixscan01.html}

Donner les défintions de: couper, embosser,tracer, poinçonner et gratter. Expliquer terme rainer.

\begin{sphinxadmonition}{attention}{Attention:}
La Curio ne découpe pas les rouleaux.
\end{sphinxadmonition}

\begin{sphinxadmonition}{attention}{Attention:}
Le poinçonnage peut être bruyant.
\end{sphinxadmonition}

\begin{sphinxadmonition}{note}{Note:}
Pour l’embossage les Silhouette Score \& Papier Relief sont recommandés.
\end{sphinxadmonition}

montrer le mode mirroir, l’envoie en découpe propose le mode mirroir normalement, idéal pour le transfert textile et le papier de transfert sur vinyle (utiliser rackclette)

Défintion: echeniller : retirer les parties négatives, celles que l’on ne souhaite pas garder.

\begin{sphinxadmonition}{note}{Note:}
le crochet Silhouette pour écheniller.
\end{sphinxadmonition}

Q: Puis\sphinxhyphen{}je découper de la matière en rouleau avec ma Curio?
A: Non. La Curio nécessite une base dure pour la découpe et possède une surface limitée de coupe.

Qu’est\sphinxhyphen{}ce que le poinçonnage?
Le poinçonnage est le processus de création d’une conception en utilisant une série de points. La Curio offre la possibilité de
produire des effets de poinçons. C’est réalisé avec un stylo feutre pour dessiner les pointillés ou avec l’outil de poinçonnage
pour marquer une série de points

Générer des templates: \sphinxurl{https://www.templatemaker.nl/fr/}

Réglages découpe 1: \sphinxurl{https://www.noscreas.fr/trucs-astuces-fiches-techniques/nos-reglages-de-decoupe/}
Réglages découpe 2: \sphinxurl{https://boutdepapier.reskator.fr/2012/04/07/reglages-de-coupe-pour-quelques-papiers/}

todo lexique tapis
Papier de transfert à aplanir pour coller la découpe
Le store et récupérer des images

Vracs:
Guide officiel: \sphinxurl{https://www.silhcdn.com/m/d/user-guides/curio-en.pdf}
Tuto: \sphinxurl{https://www.findingtimetocreate.com/2016/11/getting-started-silhouette-curio/}
Tuto 2: \sphinxurl{https://silhouette-secrets.com/2019/03/13/lets-explore-the-curio-getting-started/}
Emboss: \sphinxurl{https://youtu.be/x16kJ5ClA9I}

Setting up your Silhouette Curio: \sphinxurl{https://youtu.be/FrmaGT\_qzXo}
Curio review: \sphinxurl{https://www.youtube.com/watch?v=wg4Yd1rmpuc}

How to cut images on Silhouette Curio 3t for beginners, simple, easy, quick:
\sphinxurl{https://www.youtube.com/watch?v=HnJT4H78BOs}

\sphinxhref{http://www.silhouettefr.fr/silhouette\_tutoriels.html}{Docs}

\sphinxhref{https://www.silhouetteamerica.com/how-to}{How to}

\sphinxurl{https://youtu.be/emxLgXyxpMg}

fiche technique: \sphinxurl{https://www.lafourmicreative.fr/silhouette/86892-silhouette-curio-814792018705.html}
\begin{itemize}
\item {} 
\sphinxhref{http://silhouettefr.fr/silhouette\_curio.html}{Site officiel}

\item {} 
\sphinxhref{https://www.silhcdn.com/m/d/user-guides/curio-fr.pdf}{Documentation officiel}

\end{itemize}


\bigskip\hrule\bigskip



\section{Liens}
\label{\detokenize{tutorials/fabrication/curio/index:liens}}\begin{itemize}
\item {} 
\sphinxhref{https://youtu.be/kgOgcJDGt9c}{Tutoriels vidéos}

\item {} 
\sphinxhref{http://silhouettefr.fr/silhouette\_tutoriels.html}{Fiches tutoriels}

\item {} 
\sphinxhref{https://www.perlesandco.com/pdf/pjproduit/f/a/faq-silhouette-curio\_20160620144952.pdf}{FAQ}

\item {} 
\sphinxhref{https://www.templatemaker.nl/fr/}{Générer des formes pliables}

\end{itemize}


\section{Matériel}
\label{\detokenize{tutorials/fabrication/curio/index:materiel}}
\noindent\sphinxincludegraphics{{curio}.jpg}


\section{Logiciels}
\label{\detokenize{tutorials/fabrication/curio/index:logiciels}}\begin{itemize}
\item {} 
\sphinxhref{https://dl.silhcdn.com/58b7a26b84874c6e}{Télécharger Silhouette Studio}

\end{itemize}


\section{Utilisation}
\label{\detokenize{tutorials/fabrication/curio/index:utilisation}}

\subsection{Préparation du matériel}
\label{\detokenize{tutorials/fabrication/curio/index:preparation-du-materiel}}\begin{itemize}
\item {} 
Libérez de l’espace pour déplier les pieds de stabilisation à l’avant et arrière de la Curio.

\item {} 
Prenez une base d’adhérence aux dimensions de la matière à couper.

\item {} 
Enlevez la protection et collez la matière.

\end{itemize}

\begin{sphinxadmonition}{important}{Important:}
La matière doit être alignée en haut à droite du tapis.
\end{sphinxadmonition}
\begin{itemize}
\item {} 
Accrochez\sphinxhyphen{}le avec les quatre maintiens sur le côté.

\end{itemize}

\begin{sphinxadmonition}{attention}{Attention:}
TODO parler des plateformes et comment les choisir. Il y en a une de 1 et deux de 2. Apparemment visible dans Studio.
\end{sphinxadmonition}

\noindent\sphinxincludegraphics{{maintien}.png}
\begin{itemize}
\item {} 
Glissez le support dans la curio, l’encoche (cercle rouge) doit dépasser le bord de la Curio (trait orange):

\end{itemize}

\noindent\sphinxincludegraphics{{encoche}.png}
\begin{itemize}
\item {} 
Prenez la lame de découpe ou le stylo. Pour la lame, réglez sa profondeur avec l’avant de la Curio. La flèche rouge indique la profondeur choisie, cette flèche dit être alignée avec l’avant de la Curio pour procéder au réglage.

\end{itemize}

\begin{sphinxadmonition}{note}{Note:}
Profondeurs de lame:
\begin{itemize}
\item {} 
1: vinyl

\item {} 
2 \sphinxhyphen{} 3: papier

\item {} 
4 \sphinxhyphen{} 6: papier cartonné

\item {} 
7 \sphinxhyphen{} 10: toile

\end{itemize}
\end{sphinxadmonition}

\noindent\sphinxincludegraphics{{reglage}.png}
\begin{itemize}
\item {} 
Placez la lame ou le stylo à l’emplacement du cercle rouge, tournez l’interrupteur dans le sens inverse des aiguilles d’une montre pour l’ouvrir et dans l’autre pour le verrouiller.

\end{itemize}

Lame en place:

\noindent\sphinxincludegraphics{{position_lame}.png}

Stylo en place:

\noindent\sphinxincludegraphics{{stylo}.png}
\begin{itemize}
\item {} 
Branchez et allumez la Curio, la machine va initialiser sa position par défaut.

\end{itemize}

\begin{sphinxadmonition}{note}{Note:}
Vous pouvez initialiser la position par défaut en cliquant sur le bouton avec les deux flèches (à droite du bouton, pause).
\end{sphinxadmonition}

\noindent\sphinxincludegraphics{{boutons}.png}


\subsection{Préparer une découpe avec Silhouette Studio}
\label{\detokenize{tutorials/fabrication/curio/index:preparer-une-decoupe-avec-silhouette-studio}}
\noindent\sphinxincludegraphics{{page}.png}

a

\noindent\sphinxincludegraphics{{outils}.png}

Vous ne pouvez pas modifier la taille du texte dans les options, vous verrez plus bas comment faire.

\noindent\sphinxincludegraphics{{formatage}.png}

Après un clic:
\begin{itemize}
\item {} 
cadre de sélection pour redimensionner

\item {} 
cercle vert pour orienter

\end{itemize}

\noindent\sphinxincludegraphics{{edition}.png}


\chapter{Découpeuse vinyle: Roland GS\sphinxhyphen{}24}
\label{\detokenize{tutorials/fabrication/roland/index:decoupeuse-vinyle-roland-gs-24}}\label{\detokenize{tutorials/fabrication/roland/index::doc}}
Todo page a3

Support type piece = feuille

Maintenir le bouton Origin pour indiquer le point d’origine (le coin)

Mettre force à 80
Vitesse à 35 cm/s
Régler dans Menu, puis Condition

Cours complet Roland: \sphinxurl{https://youtu.be/i2qJ-uBZShs}

Best tuto: \sphinxurl{https://youtu.be/qFcAeMAyoUM}

\begin{sphinxadmonition}{important}{Important:}
TODO voir définition Plotter de découpe.
\end{sphinxadmonition}

Big tuto: \sphinxurl{https://youtu.be/JVDobRJxDSg}

\sphinxurl{https://youtu.be/cgqfQxDR6pU}

\sphinxurl{https://youtu.be/rDKMNXHu\_D4}

Tuto fr: \sphinxurl{https://youtu.be/5QEwoK2yzP0}

Lame la moitié de la longueur d’une carte de crédit, soit environ 2 mm.
Clockwise extends blade
Counterclockwise retracts blade


\section{Matériel}
\label{\detokenize{tutorials/fabrication/roland/index:materiel}}\begin{itemize}
\item {} 
\sphinxhref{https://www.rolanddg.fr/produits/plotters-de-decoupe/camm-1-gs-24-plotter-de-decoupe}{Site officiel}

\item {} 
\sphinxhref{https://www.machines-3d.com/images/Image/File/notice/Manuel\_utilisation\_FR\_GS24.pdf}{Manuel}

\end{itemize}

\noindent\sphinxincludegraphics{{gs24}.jpg}


\section{Logiciels}
\label{\detokenize{tutorials/fabrication/roland/index:logiciels}}\begin{itemize}
\item {} 
\sphinxhref{https://startup.rolanddg.com/RDG\_DataFiles/CAMM1/Roland\_CAMM-1\_Driver.zip}{Driver}

\end{itemize}


\section{Utilisation}
\label{\detokenize{tutorials/fabrication/roland/index:utilisation}}
\noindent\sphinxincludegraphics{{a4}.png}

\begin{sphinxadmonition}{important}{Important:}
Ne pas brancher la machine au PC si le driver n’est pas installé, sinon il sera plus possible de l’utiliser.
\end{sphinxadmonition}

\begin{sphinxadmonition}{note}{Note:}
Pour régler la langue maintenez le bouton menu et pressez le bouton d’allumage.
\end{sphinxadmonition}

Roland\_CAMM\sphinxhyphen{}1\_Driver.zip \textgreater{} RDP\sphinxhyphen{}014\_Roland\_CAMM\sphinxhyphen{}1\_Driver\_Windows10\_x64\_V150.zip \textgreater{} SETUP64.EXE

\noindent\sphinxincludegraphics{{rouleau}.png}

\noindent\sphinxincludegraphics{{lame_couverte}.png}

\noindent\sphinxincludegraphics{{lame}.png}

\noindent\sphinxincludegraphics{{lame_placée}.png}

\noindent\sphinxincludegraphics{{lame_vérouillée}.png}

\noindent\sphinxincludegraphics{{ouvert}.png}

\noindent\sphinxincludegraphics{{vérouillé}.png}


\subsection{Cut Studio}
\label{\detokenize{tutorials/fabrication/roland/index:cut-studio}}
Sur image Shift = scale proportionel

Sur la machine Pause puis Enter stop la découpe.

Fichier \textgreater{} Réglage de coupe, mettre imprimante sur Roland GS\sphinxhyphen{}24 et orientation portrait
\begin{itemize}
\item {} 
En portrait l’origine est en bas à gauche

\item {} 
En paysage l’origine est en haut à gauche

\end{itemize}

Importer BMP, PNG ou JPG, faire clic droit \textgreater{} Contour d’image…


\chapter{Papercraft: Logiciel Blender}
\label{\detokenize{tutorials/fabrication/papercraft/index:papercraft-logiciel-blender}}\label{\detokenize{tutorials/fabrication/papercraft/index::doc}}
\begin{sphinxadmonition}{note}{Note:}
Permet de réaliser des patrons pliables pour la Silhouette Curio et Roland GS\sphinxhyphen{}24.
\end{sphinxadmonition}


\chapter{Brodeuse: Bernina 500 E}
\label{\detokenize{tutorials/fabrication/bernina/index:brodeuse-bernina-500-e}}\label{\detokenize{tutorials/fabrication/bernina/index::doc}}\begin{itemize}
\item {} 
\sphinxhref{https://youtu.be/XsZuaODfv2o}{Tutoriel vidéo}

\end{itemize}


\section{Matériel}
\label{\detokenize{tutorials/fabrication/bernina/index:materiel}}
\noindent\sphinxincludegraphics{{broder}.jpg}


\section{Logiciels}
\label{\detokenize{tutorials/fabrication/bernina/index:logiciels}}

\section{Utilisation}
\label{\detokenize{tutorials/fabrication/bernina/index:utilisation}}

\chapter{Imprimante: Epson SC\sphinxhyphen{}P600}
\label{\detokenize{tutorials/fabrication/epson/index:imprimante-epson-sc-p600}}\label{\detokenize{tutorials/fabrication/epson/index::doc}}
\noindent\sphinxincludegraphics{{epson}.jpg}

Driver NA direct: \sphinxurl{https://ftp.epson.com/drivers/SCP600\_Combo\_NA.exe}

Driver NA: \sphinxurl{https://epson.com/Support/Printers/Single-Function-Inkjet-Printers/SureColor-Series/Epson-SureColor-P600/s}/SPT\_C11CE21201\#

A tester: \sphinxurl{https://www.epson.fr/products/printers/inkjet-printers/photo/surecolor-sc-p600}


\chapter{Carte: Arduino}
\label{\detokenize{tutorials/programmation/arduino/index:carte-arduino}}\label{\detokenize{tutorials/programmation/arduino/index::doc}}
\begin{sphinxadmonition}{note}{Note:}
Prendre version 1.0.6
\end{sphinxadmonition}

Tools \textgreater{} Boards \textgreater{} Intel Curie (32\sphinxhyphen{}bit) Boards \textgreater{} Arduino/Genuino 101

Niveau 1: \sphinxurl{https://openclassrooms.com/fr/courses/2778161-programmez-vos-premiers-montages-avec-arduino}

Niveau 2: \sphinxurl{https://openclassrooms.com/fr/courses/3290206-perfectionnez-vous-dans-la-programmation-arduino}

Robotique: \sphinxurl{https://openclassrooms.com/fr/courses/4076871-sinitier-a-la-robotique}

Création de schéma: \sphinxurl{https://fritzing.org/home/}

\sphinxurl{https://auth.arduino.cc/login/?challenge=eyJhbGciOiJSUzI1NiIsInR5cCI6IkpXVCJ9.eyJhdWQiOiJjdGNfcmVnaXN0cmF0aW9uIiwiZXhwIjoxNTkyODI5MTI3LCJqdGkiOiI0ZWFhNTI5Yi1iZmY5LTRiYTQtYjU3Ny03NmMzZjZjNWJhYzUiLCJyZWRpciI6Imh0dHBzOi8vaHlkcmEuYXJkdWluby5jYy9vYXV0aDIvYXV0aD9jbGllbnRfaWQ9Y3RjX3JlZ2lzdHJhdGlvblx1MDAyNnN0YXRlPW54eHY3bDhlXHUwMDI2c2NvcGU9cHJvZmlsZTpjb3JlJTIwcHJvZmlsZTpwdWJsaWMlMjBwcm9maWxlOmNvbnRhY3QlMjBwcm9maWxlOm9yZ2FuaXphdGlvbnMlMjBvcmdhbml6YXRpb246cHJpdmF0ZSUyMHByb2ZpbGU6YWRkcmVzc1x1MDAyNnJlc3BvbnNlX3R5cGU9dG9rZW5cdTAwMjZyZWRpcmVjdF91cmk9aHR0cHMlM0ElMkYlMkZjcmVhdGUuYXJkdWluby5jYyUyRmN0YyUyRnJlZyUyRiIsInNjcCI6WyJwcm9maWxlOmNvcmUiLCJwcm9maWxlOnB1YmxpYyIsInByb2ZpbGU6Y29udGFjdCIsInByb2ZpbGU6b3JnYW5pemF0aW9ucyIsIm9yZ2FuaXphdGlvbjpwcml2YXRlIiwicHJvZmlsZTphZGRyZXNzIl19.CLI79VA3LU49u0xL-ID4zPJgJXNp-pdlKs2W5BxCcqXV5dl5hUGHRiXfxDVUeWt2ylLD9U8FKJPmV3\_I5emVFJBIborvElw7\_RnEQZgSCXxcCcg6AeHWzRC-wIeBOY89MumlNgKuRSSC8MkV-OatetaaCnVu-yh-dg3xXYSjXPwLXrgXBNYTXMxwaWAFv62cVUZ\_SdkG0IZ3spAEijmR0CcX\_j8Vd6qhrRO5EXPXWR2r8QalcGLOUacCpfqvcpAm18wTZi\_qn36mPYPY6AvVRpvpIMJ8RtbfyiAp9hNF2aYZEikkVLEhKpbbqfPzPL8G2WBTyu8Rwo4TMPv1jhkEbNeIndih3oneUiXsNM9bm\_DK0VBQRP7oLg71-ziXaCRvERfQyefHS8T4q4u7MjWIWGdzgLMKvqpNHa3dzIjdubDTMYYDdics634UarZq3l-o\_I-O7QJKmNzjDxgRR5CuYDc4O12ak5Vdmw56-FiXvwKtB-GYbrOMhE0edqhlEPqYFSsbVMKfbQaBkKRm8h1Lft-ewKsAgwQgEFpJLubPPzRr8ZY8wKyg8lOgkCfj-HO4ts\_OKWrQLD-SDBPK3j4fmWHaZX7tJnUa7VYtq\_X197rMQQf9uYnlQe7CLpYYrt9HMaFhZ29LG4FHyGufuHUgNf5YewO8KVhgRh4ZaPu-LwQ}

Tuto: \sphinxurl{https://youtu.be/TV6ELrQOCcw}

Arduino 15 minutes: \sphinxurl{https://youtu.be/nL34zDTPkcs}

IDE: \sphinxurl{https://www.arduino.cc/en/main/software}

Lien direct Windows: \sphinxurl{https://www.arduino.cc/download\_handler.php}

Arduino language reference: \sphinxurl{https://www.arduino.cc/reference/en/}

Outils \textgreater{} Type de carte \textgreater{} Gestionnaire de carte, choisir Intel Curie Boards


\chapter{Carte: Micro:Bit}
\label{\detokenize{tutorials/programmation/microbit/index:carte-micro-bit}}\label{\detokenize{tutorials/programmation/microbit/index::doc}}

\chapter{Carte: Makey Makey}
\label{\detokenize{tutorials/programmation/makeymakey/index:carte-makey-makey}}\label{\detokenize{tutorials/programmation/makeymakey/index::doc}}

\section{Matériel}
\label{\detokenize{tutorials/programmation/makeymakey/index:materiel}}

\section{Logiciel}
\label{\detokenize{tutorials/programmation/makeymakey/index:logiciel}}\begin{itemize}
\item {} 
\sphinxhref{https://dist.dagoma3d.com/get/zip/CuraByDagoma/1568220765/6d41d9077db0874f55a6d89c0914a9f4}{Télécharger Cura by Dagoma} (logiciel pour préparer le modèle à l’impression)

\end{itemize}


\chapter{VR: Occulus Rift}
\label{\detokenize{tutorials/immersion/occulus/index:vr-occulus-rift}}\label{\detokenize{tutorials/immersion/occulus/index::doc}}
\begin{sphinxadmonition}{attention}{Attention:}
Le casque est mauvais pour les yeux et peut donner rapidement la nausé, les usagers doivent faire de petites sessions de 15 minutes maximum. Prévoyez éventuellement une TV en solution alternative.
\end{sphinxadmonition}


\section{Logiciels}
\label{\detokenize{tutorials/immersion/occulus/index:logiciels}}\begin{itemize}
\item {} 
\sphinxhref{https://www.oculus.com/download\_app/?id=1582076955407037}{Driver}

\end{itemize}

Todo: parler de SteamVR

\begin{sphinxadmonition}{attention}{Attention:}
Le casque VR ne permet pas un branchement en HDMI mais uniquement en display port et en mini display port avec son adaptateur.
\end{sphinxadmonition}


\section{FAQ}
\label{\detokenize{tutorials/immersion/occulus/index:faq}}

\subsection{Comment réinitialiser la vue ?}
\label{\detokenize{tutorials/immersion/occulus/index:comment-reinitialiser-la-vue}}
Sur la manette de la main doite appuyez sur le bouton central de la manette pour la main droite (bouton d’un cercle), un menu va apparaitre dans le casque, avec la manette visez le bouton « Réinitialiser la vue », appuyez sur A ou la gachette à l’arrière.
Positionnez vous à l’emplacement qui doit être celui par défaut (centré dans l’univers du jeu), gardez la tête à l’horizontal, puis appuyez sur n’importe A ou la gachette pour valider.


\subsection{Comment corriger la vidéo floue ?}
\label{\detokenize{tutorials/immersion/occulus/index:comment-corriger-la-video-floue}}
Vérifiez les éléments suivants:
\begin{itemize}
\item {} 
le casque est bien centré, s’il est légèrement trop haut ou bas par rapport à vos yeux l’image sera floue et les couleurs incorrects.

\item {} 
si vous portez des lunettes d’habitude, essayez en les portant

\item {} 
dans les options du jeu augmentez le paramètre de résolution et d’anti\sphinxhyphen{}aliasing

\end{itemize}

\begin{sphinxadmonition}{note}{Note:}
Il vaut mieux que l’image soit flou (détails graphiques faibles) et fluide, que l’inverse. Une image saccadée accentu les nausées.
\end{sphinxadmonition}


\chapter{Simulateur: Conduite}
\label{\detokenize{tutorials/immersion/conduite/index:simulateur-conduite}}\label{\detokenize{tutorials/immersion/conduite/index::doc}}\begin{itemize}
\item {} 
\sphinxhref{https://fanatec.com/eu-en/}{Site officiel}

\item {} 
\sphinxhref{https://fanatec.com/media/unknown/e6/5c/1c/Fanatec\_64-driver\_328.msi}{Drivers}

\end{itemize}


\section{Matériel}
\label{\detokenize{tutorials/immersion/conduite/index:materiel}}
\begin{sphinxadmonition}{note}{Note:}
Le matériel fonctionne sur ordinateur et sur la console Playstation 4 directement en USB.
\end{sphinxadmonition}

\noindent\sphinxincludegraphics{{sim}.png}

Le matériel mis à disposition par la MDE est composé des éléments suivants:
\begin{itemize}
\item {} 
\sphinxhref{https://www.gtomega.eu/products/apex-steering-wheel-stand}{Un cockpit pliable}

\item {} 
\sphinxhref{https://fanatec.com/eu-en/racing-wheels-wheel-bases/racing-wheels/csl-elite-racing-wheel-officially-licensed-for-ps4}{Un volant et sa base}
\begin{itemize}
\item {} 
\sphinxhref{https://fanatec.com/media/pdf/f2/2c/a0/CSL-E-Racing-Wheel-PS4-QG-FRnjUF6j1QoYeIy.pdf}{Guide de montage}

\end{itemize}

\item {} 
\sphinxhref{https://fanatec.com/eu-en/pedals/csl-elite-pedals-lc}{Trois pédales}

\item {} 
\sphinxhref{https://fanatec.com/eu-en/shifters-others/clubsport-shifter-sq-v-1.5}{Un levier de vitesse}

\end{itemize}


\chapter{Simulateur: Drones}
\label{\detokenize{tutorials/immersion/drones/index:simulateur-drones}}\label{\detokenize{tutorials/immersion/drones/index::doc}}\begin{itemize}
\item {} 
\sphinxhref{https://store.steampowered.com/app/410340/Liftoff\_FPV\_Drone\_Racing/}{Liftoff}

\end{itemize}


\chapter{Simulateur: Aerien}
\label{\detokenize{tutorials/immersion/aerien/index:simulateur-aerien}}\label{\detokenize{tutorials/immersion/aerien/index::doc}}\begin{itemize}
\item {} 
Flight Simulator 2020

\end{itemize}


\chapter{Informations pratiques}
\label{\detokenize{index:informations-pratiques}}

\section{Liens utiles}
\label{\detokenize{index:liens-utiles}}
Documentations et projets:
\begin{itemize}
\item {} 
Animations en Fablab: \sphinxurl{http://carrefour-numerique.cite-sciences.fr/fablab/wiki/doku.php?id=animations:index}

\item {} 
Carte des fablabs: \sphinxurl{https://www.makery.info/labs-map/}

\item {} 
Projets documentés (français): \sphinxurl{https://wikifab.org/wiki/Accueil}

\item {} 
Projets documentés (anglais): \sphinxurl{https://www.instructables.com/}

\item {} 
Réalisations: \sphinxurl{https://www.labboite.fr/\#!/projects}?whole\_network=f

\end{itemize}

Mémo, fil conducteur pour les ateliers

\sphinxurl{https://www.telecom-evolution.fr/fr/e-learning/parcours-fabrication-numerique-et-prototypage-rapide}
\begin{itemize}
\item {} 
Décrire comment le numérique transforme le domaine de la fabrication d’objets industriels

\item {} 
Modéliser des objets sur ordinateurs (2D et 3D)

\item {} 
Produire des objets grâce à la fabrication additive

\item {} 
Réaliser des montages simples en utilisant un micro conducteur, des capteurs et des actionneurs

\item {} 
Ecrire et exécuter du code pour programmer et connecter des objets

\item {} 
Décrire comment passer du prototype à un projet entrepreneurial

\item {} 
Appliquer des méthodes de prototypage rapide et d’innovation frugale

\end{itemize}



\renewcommand{\indexname}{Index}
\printindex
\end{document}